% THIS DOCUMENT IS TAILORED TO REQUIREMENTS FOR SCIENTIFIC COMPUTING.  IT SHOULDN'T
% BE USED FOR NON-SCIENTIFIC COMPUTING PROJECTS
\documentclass[12pt]{article}

\usepackage[square,numbers]{natbib}
\usepackage{amsmath, mathtools}
\usepackage{amsfonts}
\usepackage{amssymb}
\usepackage{graphicx}
\usepackage{colortbl}
\usepackage{xr}
\usepackage{hyperref}
\usepackage{longtable}
\usepackage{xfrac}
\usepackage{tabularx}
\usepackage{float}
\usepackage{siunitx}
\usepackage{booktabs}
\usepackage{caption}
\usepackage{pdflscape}
\usepackage{afterpage}

\usepackage[round]{natbib}

%\usepackage{refcheck}

\hypersetup{
    bookmarks=true,         % show bookmarks bar?
      colorlinks=true,       % false: boxed links; true: colored links
    linkcolor=red,          % color of internal links (change box color with linkbordercolor)
    citecolor=green,        % color of links to bibliography
    filecolor=magenta,      % color of file links
    urlcolor=cyan           % color of external links
}



% For easy change of table widths
\newcommand{\colZwidth}{1.0\textwidth}
\newcommand{\colAwidth}{0.13\textwidth}
\newcommand{\colBwidth}{0.82\textwidth}
\newcommand{\colCwidth}{0.1\textwidth}
\newcommand{\colDwidth}{0.05\textwidth}
\newcommand{\colEwidth}{0.8\textwidth}
\newcommand{\colFwidth}{0.17\textwidth}
\newcommand{\colGwidth}{0.5\textwidth}
\newcommand{\colHwidth}{0.28\textwidth}

% Used so that cross-references have a meaningful prefix
\newcounter{defnum} %Definition Number
\newcommand{\dthedefnum}{GD\thedefnum}
\newcommand{\dref}[1]{GD\ref{#1}}
\newcounter{datadefnum} %Datadefinition Number
\newcommand{\ddthedatadefnum}{DD\thedatadefnum}
\newcommand{\ddref}[1]{DD\ref{#1}}
\newcounter{theorynum} %Theory Number
\newcommand{\tthetheorynum}{T\thetheorynum}
\newcommand{\tref}[1]{T\ref{#1}}
\newcounter{tablenum} %Table Number
\newcommand{\tbthetablenum}{T\thetablenum}
\newcommand{\tbref}[1]{TB\ref{#1}}
\newcounter{assumpnum} %Assumption Number
\newcommand{\atheassumpnum}{P\theassumpnum}
\newcommand{\aref}[1]{A\ref{#1}}
\newcounter{goalnum} %Goal Number
\newcommand{\gthegoalnum}{P\thegoalnum}
\newcommand{\gsref}[1]{GS\ref{#1}}
\newcounter{instnum} %Instance Number
\newcommand{\itheinstnum}{IM\theinstnum}
\newcommand{\iref}[1]{IM\ref{#1}}
\newcounter{reqnum} %Requirement Number
\newcommand{\rthereqnum}{P\thereqnum}
\newcommand{\rref}[1]{R\ref{#1}}
\newcounter{nfrnum} %NFR Number
\newcommand{\rthenfrnum}{NFR\thenfrnum}
\newcommand{\nfrref}[1]{NFR\ref{#1}}
\newcounter{lcnum} %Likely change number
\newcommand{\lthelcnum}{LC\thelcnum}
\newcommand{\lcref}[1]{LC\ref{#1}}

\usepackage{fullpage}

\newcommand{\deftheory}[9][Not Applicable]
{
\newpage
\noindent \rule{\textwidth}{0.5mm}

\paragraph{RefName: } \textbf{#2} \phantomsection 
\label{#2}

\paragraph{Label:} #3

\noindent \rule{\textwidth}{0.5mm}

\paragraph{Equation:}

#4

\paragraph{Description:}

#5

\paragraph{Notes:}

#6

\paragraph{Source:}

#7

\paragraph{Ref.\ By:}

#8

\paragraph{Preconditions for \hyperref[#2]{#2}:}
\label{#2_precond}

#9

\paragraph{Derivation for \hyperref[#2]{#2}:}
\label{#2_deriv}

#1

\noindent \rule{\textwidth}{0.5mm}

}

\begin{document}

\title{Software Requirements Specification for Truss Tool: 
A Tool for Truss Analysis} 
\author{Maryam Valian}
\date{\today}
	
\maketitle

~\newpage

\pagenumbering{roman}

\tableofcontents

~\newpage

\section*{Revision History}

\begin{tabularx}{\textwidth}{p{3cm}p{2cm}X}
\toprule {\bf Date} & {\bf Version} & {\bf Notes}\\
\midrule
 2023-01-30 & 1.0 & Initial version of the SRS\\
 2023-01-10& 1.1 & Modification according to the feedback\\
 2023-02-20 & 1.2 & Updating according to SRSfdbk.pdf\\
 2023-03-23 & 1.3 & Minor update in Table of symbols.\\ 
\bottomrule
\end{tabularx}

~\newpage

\section{Reference Material}

This section records information for easy reference.

\subsection{Table of Units}

Throughout this document, SI (Syst\`{e}me International d'Unit\'{e}s) is employed
as the unit system.  In addition to the basic units, several derived units are
used as described below.  For each unit, the symbol is given followed by a
unit description and the SI name.
~\newline

\renewcommand{\arraystretch}{1.2}
%\begin{table}[ht]
  \noindent \begin{tabular}{l l l} 
    \toprule		
    \textbf{symbol} & \textbf{unit} & \textbf{SI}\\
    \midrule 
    \si{\metre} & length & metre\\
    \si{\newton} & force & newton\\
    \si{\deg} & angle & degree\\
        \bottomrule
  \end{tabular}
  %	\caption{Provide a caption}
%\end{table}

\subsection{Table of Symbols}

The table that follows summarizes the symbols used in this document along with
their units.  The choice of symbols was made to be consistent with the structural statics literature and with existing documentation for the truss analysis problem. The symbols are listed in alphabetical order.

\renewcommand{\arraystretch}{1.2}
%\noindent \begin{tabularx}{1.0\textwidth}{l l X}
\noindent \begin{longtable*}{l l p{12cm}} \toprule
\textbf{symbol} & \textbf{unit} & \textbf{description}\\
\midrule 
$F_\text{i}$ & \si {\newton} & External force of joint i \\
$F_\text{xi}$ & \si{\newton} & Force component in the x direction of joint i \\
$F_\text{yi}$ & \si{\newton} & Force component in the y direction of joint i \\
$I_\text{m}$ & \si{\newton} & Internal force of member $m$\\
$M_\text{i}$ & \si{\newton}\si{\metre} & Moment component of joint i \\
$P_\text{x}$ & - & Pin support reaction in X-axis direction \\
$P_\text{y}$ & - & Pin support reaction  in Y-axis direction \\
$R_\text{y}$ & - & Roller support reaction in Y-axis direction \\
$S_\text{p}$ & - & Joint index of Pin support \\
$S_\text{r}$ & - & Joint index of Roller support \\
$\theta$ & \si{\deg} & Angle between members and x-axis  \\

\bottomrule
\end{longtable*}


\subsection{Abbreviations and Acronyms}

\renewcommand{\arraystretch}{1.2}
\begin{tabular}{l l} 
  \toprule		
  \textbf{symbol} & \textbf{description}\\
  \midrule 
  A & Assumption\\
  DD & Data Definition\\
  GD & General Definition\\
  GS & Goal Statement\\
  IM & Instance Model\\
  LC & Likely Change\\
  PS & Physical System Description\\
  R & Requirement\\
  SRS & Software Requirements Specification\\
 
  T & Theoretical Model\\
  \bottomrule
\end{tabular}\\



\subsection{Mathematical Notation}
In this document, we do not use any specific mathematical notation.

\newpage

\pagenumbering{arabic}

\section{Introduction}
{A truss is a structure that consists of members organized into connected triangles to enable the distribution of loads and forces. Trusses are most commonly used for wide spans like bridges, and roofs. Figure \ref{Fig_bridge} shows a bridge built with planar trusses in 1969 called the Burlington lift bridge. Truss Analysis shows whether the external forces are well-distributed among the members or not. \\
We introduce Truss Tool that helps engineers analyze a given planar truss. The following section provides an overview of the Truss Tool's software requirements specification (SRS). In this section, first, we explain the purpose of the document. Then we describe the scope of the requirements, the intended reader's characteristics and the document's organization.
}

\begin{figure}[h!]
\begin{center}
 \includegraphics[width=0.6\textwidth]{bridge.jpg}
\caption{Burlington lift bridge: an example of the planar truss}
\label{Fig_bridge} 
\end{center}
\end{figure}

\subsection{Purpose of Document}
{The primary purpose of this document is to outline the software requirements of the truss analysis tool. To provide a good understanding of the system, different aspects of the system such as  goals, assumptions, theoretical models, and definitions will be explained.  The following SRS document will remain abstract exploring what is being solved rather than how it will be solved.\\
The following document will describe the system context and constraints, the specific problem definition and solution characteristics, requirements and likely and unlikely changes for the 
development of the tool.}

\subsection{Scope of Requirements} 
{The scope of the requirements includes the analysis of the two-dimensional trusses where all members and nodes lie within a two-dimensional plane. For more details, you can also see the assumptions section  (Section~\ref{sec_assumpt}). }

\subsection{Characteristics of Intended Reader} \label{sec_IntendedReader}
Reviewers of this documentation should have a basic understanding of software development and structure statics and high school physics and high school Mathematics.  The users of the Truss Tool must have a higher level of expertise, as explained in Section: User Characteristics (Section~\ref{SecUserCharacteristics}). 

\subsection{Organization of Document}
The organization of the document follows the template for an SRS for scientific computing software proposed by \citet{SmithandLai2005}. The template will present the system's goals, theories, definitions, and assumptions. Readers interested in top-down reading can begin by reading the system's goal statements (Section~\ref{Sec_gs}). Subsequently, the theoretical models will elaborate on the goal statements. Lastly, readers can finish with a more  understanding of the system by reading instance models of the system.

\section{General System Description}

This section provides general information about the system.  It identifies the interfaces between the system and its environment, describes the user characteristics and lists the system constraints. 

\subsection{System Context}
Figure~\ref{Fig_SystemContext} shows the system context. The circles 
represent a user that interacts with the software. The rectangle represents the software system for the truss tool. The arrows display the input data from the user and the output data that is useful for the user.

\begin{figure}[h!]
\begin{center}
 \includegraphics[width=0.8\textwidth]{systemcontext .jpg}
\caption{System Context}
\label{Fig_SystemContext} 
\end{center}
\end{figure}

The interaction between the product and the user is through a user interface. The responsibilities of the user and the system are as follows:
\begin{itemize}
\item User Responsibilities:
\begin{itemize}
\item Provide truss geometry, supports and external forces. 
\item Ensure the input data describes a valid truss.

\end{itemize}
\item Truss Tool Responsibilities:
\begin{itemize}
\item Detect data type mismatch, such as a string of characters instead of a  floating point number.
\item Calculate external forces and support's reaction.
\end{itemize}
\end{itemize}


\subsection{User Characteristics} \label{SecUserCharacteristics}

The end user of Truss Tool should be an architecture/civil/mechanic engineer or should have an understanding of undergraduate Level 1 structural analysis.

\subsection{System Constraints}

There are no constraints on the development of the Truss Tool.


\section{Specific System Description}

This section first presents the problem description, which gives a high-level view of the problem to be solved.  This is followed by the solution characteristics specification, which presents the assumptions, theories, definitions and finally the instance models.


\subsection{Problem Description} \label{Sec_pd}

Truss Tool is intended to solve a given truss with given external forces. By solving a truss, we mean that we are interested to calculate all internal forces among the members and the reactions of the supports. As a result, Truss Tool will help engineers to make a decision on whether the design of the given truss is proper or not.


\subsubsection{Terminology and  Definitions}


This subsection provides a list of terms that are used in the subsequent
sections and their meaning, with the purpose of reducing ambiguity and making it easier to correctly understand the requirements:

\begin{itemize}

\item{ Planar truss:  A planar truss is one where all members and nodes lie within a two-dimensional plane.} 
\item{Joint (nodes): A place where two or more  members of the truss  are connected.}
\item{Force equilibrium: A body is in force equilibrium if the sum of all the forces acting on the body is zero.}
\item{Moment equilibrium: A body is in moment equilibrium if the sum of all the moments of all the forces acting on the body is zero.}
\item{Moment: The turning effect of a force is called the moment. The moment is the result of the force multiplied by the perpendicular distance from the line of action of the force to the pivot or point where the object will turn.}
\item{Compression: The internal force acts to shorten the member.}
\item{Tension: The internal force acts to long the member.}
\item{Pinned support: A pinned support can resist both vertical and horizontal forces but not a moment. }
\item{Roller support: Roller support and reactions resist perpendicular forces and they cannot resist parallel or horizontal forces and moments. It means the roller support will move freely along the surface without resisting horizontal force. }
\item{Zero force members: Members which do not have any force in them.}
\end{itemize}

\subsubsection{Physical System Description} \label{sec_phySystDescrip}

The physical system of the Truss Tool, as shown in Figure~\ref{fig_physys},
includes the following elements:

\begin{itemize}

\item{PS1: The joints ($j_\text{1}$,$j_\text{2}$,..,$j_\text{n}$).}
\item{PS2: The members ($m_\text{1}$,$m_\text{2}$,..,$m_\text{k}$).}
\item{PS3: The supports ($S_\text{1}$,$S_\text{2}$).}

\end{itemize}

 \begin{figure}[h!]
\begin{center}
 %\rotatebox{-90}
 
  \includegraphics[width=0.7\textwidth]{physic_system.jpg}
 
 \caption{The physical system of Truss Tool}
 \label{fig_physys}
 \end{center}
 \end{figure}

\subsubsection{Goal Statements} \label{Sec_gs}

\noindent Given the truss features and external forces, the goal statements are:

\begin{itemize}
\item{GS\refstepcounter{goalnum}\thegoalnum \label{G_react}: Calculate the reactions of the supports.}
\item{GS\refstepcounter{goalnum}\thegoalnum \label{G_force}: Calculate the internal forces for each member.}
\end{itemize}

\subsection{Solution Characteristics Specification}
The instance models that govern Truss Tool are presented in
Subsection~\ref{sec_instance}.  The information to understand the meaning of the instance models and their derivation is also presented so that the instance models can be verified.

\subsubsection{Assumptions} \label{sec_assumpt}

This section simplifies the original problem and helps in developing the
theoretical model by filling in the missing information for the physical
system. The numbers given in the square brackets refer to the theoretical model
[T], the general definition [GD], data definition [DD], instance model [IM], or
likely change [LC], in which the respective assumption is used.

\begin{itemize}

\item{A\refstepcounter{assumpnum}\theassumpnum \label{planar}: All members and nodes lie within a two-dimensional plane.}

\item{A\refstepcounter{assumpnum}\theassumpnum \label{connection}: Members are inter-connected only at their ends.}
\item{A\refstepcounter{assumpnum}\theassumpnum \label{as_straight}: Members must be straight.}
\item{A\refstepcounter{assumpnum}\theassumpnum \label{frictionless}: All joints are frictionless hinges.}
\item{A\refstepcounter{assumpnum}\theassumpnum \label{Force_at_joints}: All forces must only be applied at joints}
\item{A\refstepcounter{assumpnum}\theassumpnum \label{reaction_at_joints}: All reactions must only be applied at joints}
\item{A\refstepcounter{assumpnum}\theassumpnum \label{self_w}: Self-weight of the member will be neglected}
\item{A\refstepcounter{assumpnum}\theassumpnum \label{axial_fmem}: Members are subjected to axial forces only.}
\item{A\refstepcounter{assumpnum}\theassumpnum \label{maxsupport}: The number of supports are at most two.}

\end{itemize}

\subsubsection{Theoretical Models}\label{sec_theoretica}

This section focuses on the general equations and laws that Truss Tool is based on. 

%----------------------------TM1
~\newline

\noindent
\begin{minipage}{\textwidth}
\renewcommand*{\arraystretch}{1.5}
\begin{tabular}{| p{\colAwidth} | p{\colBwidth}|}
\hline
\rowcolor[gray]{0.9}
Number& T\refstepcounter{theorynum}\thetheorynum \label{equil}\\
\hline
Label &\bf General Equilibrium equations \\
\hline
SI Units& All Forces are measured in \si{\newton} \\
& Moments are measured in \si{\newton\metre}\\
\hline
Equation& $\sum F_{x}=0 $, $\sum F_{y}=0 $, $\sum M=0$\\
\hline
Description &
The equilibrium equation describes the static equilibrium of all  forces of the system and the moment for the system so that $\sum M =0$ and $\sum F=0$.\\
  & $F$ is any force in the system (\si{\newton}). $M$ is a moment that is the turning effect of a force. Moments act about a point in a clockwise or anticlockwise direction(\si{\newton\metre})\\
 & According to \aref{planar} the summation of forces either horizontal $\sum F_{x}=0 $ or vertical $\sum F_{y}=0 $ should be equal to zero.
\\
\hline
  Source & \cite{hibbeler2006structural} \\
  \hline
  Ref.\ By & \iref{Supp_react},\iref{i_force}\\
  \hline
\end{tabular}
\end{minipage}\\

\newpage
~\newline

\subsubsection{General Definitions}\label{sec_gendef}

Not applicable.

\subsubsection{Data Definitions}\label{sec_datadef}

This section collects and defines all the data needed to build the instance models. The dimension of each quantity is also given. 

~\newline

\noindent
\begin{minipage}{\textwidth}
\renewcommand*{\arraystretch}{1.5}
\begin{tabular}{| p{\colAwidth} | p{\colBwidth}|}
\hline
\rowcolor[gray]{0.9}
Number& DD\refstepcounter{datadefnum}\thedatadefnum \label{length_mem}\\
\hline
Label& \bf Length of a straight Line \\
\hline
Symbol &$L(m): Member \rightarrow \mathbb{R}$\\
\hline

  SI Units & \si{\metre}\\
  \hline
  Equation&$L = \sqrt{(x_{2}-x_{1})^2+(y_{2}-y_{1})^2}$\\
 \hline
Description & 
   For every two points such as $X_{1}$,$X_{2}$   with coordination $(x_{1},y_{1})$ and $(x_{2},y_{2})$ the length of line between two point is $L$. \\
  \hline
  Sources& \url{https://www.cuemath.com/distance-formula/} \\
  \hline
  Ref.\ By & \iref{Design_truss}\\
  \hline
\end{tabular}
\end{minipage}\\
%------------------------------------DD2
~\newline


\noindent
\begin{minipage}{\textwidth}
\renewcommand*{\arraystretch}{1.5}
\begin{tabular}{| p{\colAwidth} | p{\colBwidth}|}
\hline
\rowcolor[gray]{0.9}
Number& DD\refstepcounter{datadefnum}\thedatadefnum \label{Law_cos}\\
\hline
Label& \bf Finding angle by Law of cosine \\
\hline
Symbol &$\theta$\\
\hline

  SI Units & Degree\\
  \hline
  Equation& $\theta= \arccos(\frac{a^2+b^2+c^2}{2ab}$)\\
 \hline
Description & 
   The Law of Cosine helps us to find any angle for a  given triangle with a known length of sides. Where $\theta$ is the angle between sides $a$ and $b$. and $c$ is the length of the opposite side. \\
  \hline
  Sources& \url{https://en.wikipedia.org/wiki/Law_of_cosines} \\
  \hline
  Ref.\ By & \iref{Design_truss}\\
  \hline
\end{tabular}
\end{minipage}\\

%-------------------------------DD3
~\newline


\noindent
\begin{minipage}{\textwidth}
\renewcommand*{\arraystretch}{1.5}
\begin{tabular}{| p{\colAwidth} | p{\colBwidth}|}
\hline
\rowcolor[gray]{0.9}
Number& DD\refstepcounter{datadefnum}\thedatadefnum \label{decomp}\\
\hline
Label& \bf Decomposition of the force \\
\hline
Symbol &$D(F,\theta) \rightarrow (F_x,F_y)$\\
\hline

  SI Units & (\si{\newton , \newton})\\
  \hline
  Equation& $F_{x}=F\cos{\theta}$ , $F_{y}=F\sin{\theta}$\\
 \hline
Description & 
   Each force say F with angle $\theta$ with horizontal line can be decomposed to $F_x$ and $F_y$ \\
  \hline
  Sources& \cite{galishnikova2009geometrically} \\
  \hline
  Ref.\ By & \iref{Design_truss}\\
  \hline
\end{tabular}
\end{minipage}\\
\subsubsection{Data Types}\label{sec_datatypes}

This section collects and defines all the data types needed to document the models. For Truss Tool, all data types are real numbers or Boolean.

\subsubsection{Instance Models} \label{sec_instance}    

This section transforms the problem defined in Section~\ref{Sec_pd} into 
one which is expressed in mathematical terms. It uses concrete symbols defined 
in Section~\ref{sec_datadef} to replace the abstract symbols in the models 
identified in Sections~\ref{sec_theoretica}.

The goals \gsref{G_react} and \gsref{G_force} are solved by \iref{Design_truss}, \iref{Supp_react}, \iref{i_force}
 
~\newline

\subsubsection{Data Types}\label{sec_datatypes}

This section collects and defines all the data types needed to document the models. For Truss Tool, all data types are real or integer numbers.

\subsubsection{Instance Models} \label{sec_instance}    

For better understanding the inputs and outputs of the instant models consider a simple truss with $n=3$ joints as shown in figure \ref{fig_example}. The other trusses with higher number of joints can modeled similarly.
\begin{figure}[h!]
\begin{center}
 %\rotatebox{-90}
 
  \includegraphics[width=0.5\textwidth]{example.jpg}
 
 \caption{Example of simple truss for better understanding.}
 \label{fig_example}
 \end{center}
 \end{figure}

The input for \iref{Design_truss} is geometric location of each joint $J=\{(0,0),(1,1),(2,0)\}$. Also we have index for joint, for example $J_{1} =(0,0)$, $J_{2} =(1,1)$, $J_{3} =(2,0)$. The other input is Members $M=\{(1,2),(1,3),(2,3)\}$. for example $M_{1} =(1,2)$ which means that $J_{1}$ is connected to $J_{2}$. Now by applying \ddref{length_mem} we can calculate 
$L_{1} =\sqrt{(1-0)+(1-0)} = 1.41 $. Same for $M_{2} =(1,3)$, the distance between $j_{1} =(0,0)$ and $J_{3} =(2,0)$ is calculated as $2$. The out put of the \iref{Design_truss} is 
$L=\{1.41,2,1.41$\}$
$By \ddref{Law_cos} and $L$ we can calculate $\theta$ for each joint. $\theta_{1} = \arccos(\frac{(1.41)^2-(1.41)^2-(2)^2}{-2*2*1.41} = 45$. The array of angles will be calculated as $\theta =\{ 45,90,45\}$. For \iref{Supp_react} other inputs are modeled as external forces on each joint $F=\{(0,0),(0,-50),(0,0)\}$ which means there is an external force only on $J_{2}$. Also $S_p =1 $ and $S_r =3 $ which means pinned support is on $J_{1} = (0,0)$ and roller support is on $J_{3} =(2,0)$. The output will be calculated as $R_{px} =0, R_{py}=+25, R_{ry}=+25$. 

 
~\newline


%--------------------Instance Model 1

\noindent
\begin{minipage}{\textwidth}
\renewcommand*{\arraystretch}{1.5}
\begin{tabular}{| p{\colAwidth} | p{\colBwidth}|}
  \hline
  \rowcolor[gray]{0.9}
  Number& IM\refstepcounter{instnum}\theinstnum \label{Design_truss}\\
  \hline
  Label& \bf Calculate truss design features \\
  \hline
  Input& $J_{i}=$ Tuples of $(X,Y)$ location of joints \\& $M_{j}=$ Tuples of end-joints for the members where each end-joint has a specific integer index. \\
 
   \hline
   Output& $L_{j}=$ Tuples of the length of all members such that $0 \leq L_{j} \leq 20$ \\
   &$\theta_{p,q}= $ Tuples of all angles between two members $M_{p}$ and $M_{q}$ so that $0<\theta<180$ \\
   \hline
  Description&  For each member, Truss Tool should calculate the length $L{j}$ from \ddref{length_mem}. For each two members $M_{p}$ and $M_{q}$, Truss Tool should calculate $\theta_{p,q}$ from \ddref{Law_cos}.\\
  
 \hline
  Sources& \cite{galishnikova2009geometrically} \\
  \hline
  Ref.\ By & \iref{i_force}\\
  \hline
\end{tabular}
\end{minipage}\\
~\newline

%----------------------Instance Model 2-------------------
\noindent
\begin{minipage}{\textwidth}
\renewcommand*{\arraystretch}{1.5}
\begin{tabular}{| p{\colAwidth} | p{\colBwidth}|}
  \hline
  \rowcolor[gray]{0.9}
  Number& IM\refstepcounter{instnum}\theinstnum \label{Supp_react}\\
  \hline
  Label& \bf Find support's reactions \\
  \hline
  Input&   $S_{p}=$ Position of the pinned support as joint index, $S_{r}$ Position of the roller support as joint index\\
  & $F_{i}=$ Tuples of a joint index for the position of an external force, and the amount of force in $x$ direction and $y$ direction (\si{\newton}) \\ 

  \hline
  Output& $R_px$, $R_py$ for pinned support ($S_{p}$) and $R_ry$ for roller support ($S_{r}$) \\

  \hline
  Description & By considering the whole truss as a free body, The reaction of the supports should be calculated from \tref{equil}\\
 & The input is constrained so that we have at most only two supports at most $S_p , S_r$ (\aref{maxsupport})\\
   & For inputting the position of an External force or support, the index of a join is needed. (\aref{Force_at_joints},\aref{reaction_at_joints}) \\
 \hline
  Sources& \cite{galishnikova2009geometrically} \\
  \hline
  Ref.\ By & \iref{i_force},\gsref{G_react}\\
  \hline
\end{tabular}
\end{minipage}\\

~\newline

%----------------------------Instance Model 3

\noindent
\begin{minipage}{\textwidth}
\renewcommand*{\arraystretch}{1.5}
\begin{tabular}{| p{\colAwidth} | p{\colBwidth}|}
  \hline
  \rowcolor[gray]{0.9}
  Number& IM\refstepcounter{instnum}\theinstnum \label{i_force}\\
  \hline
  Label& \bf Calculate internal forces for all members  $IF_{m}$\\
  \hline
  Input& $F_{i}=$ External forces. External forces can be applied only at joints. We have index of each joint as integer number i. Amount of external force should be entered as $F_x,F_y$.\\
  & $L{j}$ all Members lengths, $\theta_{p,q}$ all angles between members from \iref{Design_truss}\\
  & $R_px$,$R_py$, $R_ry$  reactions from \iref{Supp_react}\\
  \hline
   Output& $IF_{j}$ Internal force for each member by solving ($n-1$) equation of equilibrium for each joint except the last joint. where $j$ is an integer index for a member.(\si{\newton})\\
  \hline
  Description& By decomposition of each internal force $IF_{i}$ to $IF_{x}$ and $IF_{y}$ (\aref{planar}) and applying equilibrium equations from \tref{equil} for each joint, the internal forces will be calculated jointly. For the last joint $j_{n}$, there will be no unknown internal force left. Hence we can use the last equation as the verification test of output correctness.  \\
 
 \hline
  Sources& \cite{galishnikova2009geometrically} \\
  \hline
  Ref.\ By & \gsref{G_force}\\
  \hline
\end{tabular}
\end{minipage}\\
~\newline
%-----------------------------
\subsubsection{Input Data Constraints} \label{sec_DataConstraints}    

Table~\ref{TblInputVar} shows the data constraints on the input-output
variables.  The column for physical constraints gives the physical limitations
on the range of values that the variable can take.  The column for software constraints restricts the range of inputs to reasonable values.  The software constraints will be helpful in the design stage for picking suitable algorithms.  The constraints are conservative, allowing the model user to experiment with unusual situations.  The column of typical values is intended to provide a feel for a common scenario.  The uncertainty column estimates the confidence with which the physical quantities can be
measured.  This information would be part of the input if one were performing an
uncertainty quantification exercise.\\
The specification parameters in Table~\ref{TblInputVar} are listed in
Table~\ref{TblSpecParams}.

\begin{table}[!h]
  \caption{Input Variables} \label{TblInputVar}
  \renewcommand{\arraystretch}{1.2}
\noindent \begin{longtable*}{l l l l c} 
  \toprule
  \textbf{Var} & \textbf{Physical Constraints} & \textbf{Software Constraints} &
                             \textbf{Typical Value} & \textbf{Uncertainty}\\
  \midrule 
  $n$ & $n > 3$ & $n_{\text{min}} \leq n \leq n_{\text{max}}$ & 5  & 5\%
  \\
  \bottomrule
\end{longtable*}
\end{table}

\noindent 
\begin{description}

\item[(*)] The count of Joints in a given truss is an integer number. it must be greater or equal to 3 to be considered a triangle. For small trusses, the number of joints is around 8. The maximum number of joints $n_{max}$ for the run time considerations will be considered 20.
\end{description}

\begin{table}[!h]
\caption{Specification Parameter Values} \label{TblSpecParams}
\renewcommand{\arraystretch}{1.2}
\noindent \begin{longtable*}{l l} 
  \toprule
  \textbf{Var} & \textbf{Value} \\
  \midrule 
  $n_\text{min}$ & 3 \\
  $n_\text{max}$ & 9\\
  \bottomrule
\end{longtable*}
\end{table}

\subsubsection{Properties of a Correct Solution} \label{sec_CorrectSolution}

\noindent
Table~\ref{TblOutputVar} shows the physical constraints on the output. Suppose all joints from index 1 to $n-1$ are solved. Then all $IF_{m}$ are already calculated and the last joint will be considered as a physical constraint on the output.

\begin{table}[!h]
\caption{Output Variables} \label{TblOutputVar}
\renewcommand{\arraystretch}{1.2}
\noindent \begin{longtable*}{l l} 
  \toprule
  \textbf{Var} & \textbf{Physical Constraints} \\
  \midrule 
  $\sum F_{n}=0$ &  (by~\aref{planar},\aref{Force_at_joints})
  \\
  \bottomrule
\end{longtable*}
\end{table}

\section{Requirements}

This section provides the functional requirements and the business tasks that the
software is expected to be complete, and the nonfunctional requirements, the
qualities that the software is expected to exhibit.

\subsection{Functional Requirements}

\noindent \begin{itemize}

\item{R\refstepcounter{reqnum}\thereqnum \label{R_Inputs}: Input the values from Table \ref{TblInputVal}}

\item{R\refstepcounter{reqnum}\thereqnum \label{R_OutputInputs}:Echoing inputs as part of output.} 

\item{R\refstepcounter{reqnum}\thereqnum \label{R_Calculate}: Calculate support reactions from \iref{Supp_react}, and internal forces from \iref{Design_truss}, \iref{Supp_react}, \iref{i_force}}

\item{R\refstepcounter{reqnum}\thereqnum \label{R_VerifyOutput}: Check summation of internal forces is zero in the last joint. \iref{i_force}}
 \end{itemize}
\begin{table}[!h]
	\caption{Required Inputs} \label{TblInputVal}
	\renewcommand{\arraystretch}{1.2}
	\noindent \begin{longtable*}{l l l l c} 
		\toprule
		\textbf{Symbol} & \textbf{Description} & \textbf{Data Type} \\
		\midrule 
		$F$  & External forces on each joint Index  & Array of Integer (\si{\newton}). \\
		$J$  & The location of all joint & Array of Real tuples \\
		$M$  & A pair of joint index for all members &Array of Integer tuples \\
		
		$S_1, S_2$ & pair of Joint Index as Location of supports and type of support   &(Integer, Char) \\
		\bottomrule
		
	\end{longtable*}
\end{table}

\subsection{Nonfunctional Requirements}



\noindent \begin{itemize}


 
\item[NFR\refstepcounter{nfrnum}\thenfrnum \label{NFR_Usability}:] \textbf{Reliability: The Time performance of the software should be able to tested through
verification and validation plan (VnV Plan).}
 

\item[NFR\refstepcounter{nfrnum}\thenfrnum \label{NFR_Maintainability}:]
  \textbf{Maintainability: The effort required to make any of the likely    changes listed for Truss Tool should be less than 30\% of the original    development time.}

\item[NFR\refstepcounter{nfrnum}\thenfrnum \label{NFR_Portability}:]
  \textbf{Portability:  Truss Tool is runnable on Windows.}

\end{itemize}

\section{Likely Changes}    

\noindent \begin{itemize}

\item[LC\refstepcounter{lcnum}\thelcnum\label{LC_planar}:] The software may be changed to solve both types of trusses: two-dimensional and three-dimensional [\aref{planar}]  
\item[LC\refstepcounter{lcnum}\thelcnum\label{LC_friction}:] The software may be changed to consider friction of the joints [\aref{frictionless} ]

\end{itemize}

\section{Unlikely Changes}    

\noindent \begin{itemize}

\item[UC1\label{UC_connection}:] The truss members are only connected at their joints [\aref{connection}]
\item[UC2\label{UC_straight}:] The truss members are straight [\aref{as_straight}]
\end{itemize}

\section{Traceability Matrices and Graphs}

The purpose of the traceability matrices is to provide easy references on what
has to be additionally modified if a certain component is changed.  Every time a
component is changed, the items in the column of that component that is marked
with an ``X'' may have to be modified as well.  Table~\ref{Table:trace} shows the
dependencies of theoretical models, general definitions, data definitions, and
instance models with each other. Table~\ref{Table:R_trace} shows the
dependencies of instance models, requirements, and data constraints on each
other. Table~\ref{Table:A_trace} shows the dependencies of theoretical models,
general definitions, data definitions, instance models, and likely changes in
the assumptions.

\begin{table}[h!]
\centering
\begin{tabular}{|c|c|c|c|c|c|c|}
\hline
	& \tref{equil} &\ddref{length_mem}  &  \ddref{Law_cos}
  & \ddref{decomp} &\iref{Design_truss}& \iref{Supp_react} 
	  \\
\hline
\tref{equil}       & & & & & & \\ \hline
 \ddref{length_mem}   &X & & & & & \\ \hline
  \ddref{Law_cos}  & & & & & & \\ \hline
  \ddref{decomp}     & & & & & & \\ \hline
\iref{Design_truss} & & &X &X & & \\ \hline 
\iref{Supp_react} & &X & & & & \\ \hline 
\iref{i_force} & &X & & &X &X \\ \hline 
\end{tabular}
\caption{Traceability Matrix Showing the Connections Between  theoretical models, general definitions,
data definitions and Instance Models}
\label{Table:trace}
\end{table}
%----------------
\begin{table}[h!]
\centering
\begin{tabular}{|c|c|c|c|c|}
\hline
	& \iref{Design_truss}& \iref{Supp_react}& 
	\iref{i_force}& Constraint \ref{sec_DataConstraints}  \\
\hline
R1      &X & & &X\\ \hline
R2    &X & &  &X\\ \hline
R3    &X &X &X&\\ \hline
R4       & & & &X\\ \hline
\end{tabular}
\caption{Traceability Matrix Showing the Connections Between Requirements and Instance Models}
\label{Table:R_trace}
\end{table}
%--------------------
\begin{table}[h!]
\centering
\begin{tabular}{|c|c|c|c|c|c|c|c|c|c|c|}
\hline
	& \aref{planar}& \aref{connection}& \aref{as_straight}  &\aref{frictionless} & \aref{Force_at_joints} & \aref{reaction_at_joints} & \aref{self_w} 
 & \aref{axial_fmem} &\aref{maxsupport}\\
\hline
\tref{equil}       &X & & & & & & & & \\ \hline
  \ddref{length_mem}    & & &X & & & & & & \\ \hline
  \ddref{Law_cos}     & & &X &X & & & & & \\ \hline
  \ddref{decomp}   &X &X &X & & & & & & \\ \hline
\iref{Design_truss}      & &X &X & & & & & & \\ \hline
\iref{Supp_react}      &X &X &X &X &X &X & & &X \\ \hline
\iref{i_force}      &X & & &X & & &X &X &X \\ \hline
\end{tabular}
\caption{Traceability Matrix Showing the Connections Between assumptions and other sections. For example \aref{as_straight} is essential to calculate distance by\ddref{length_mem}}
\label{Table:A_trace}
\end{table}


\section{Development Plan}

Not applicable.


\section{Values of Auxiliary Constants}
\begin{table}[h!]
	\renewcommand{\arraystretch}{1.2}
	\noindent \begin{longtable*}{l l l l c} 
		\toprule
		\textbf{Symbol} & \textbf{Description} & \textbf{Value} & 
		\textbf{Units} \\
		\midrule 
		$F_{\text{max}}$ & the maximum value for external force & 35000 & 
		\si{\newton}\\
		$F_{\text{min}}$ & the minimum value for external force & -35000 & 
		\si{\newton}\\
       	
		$\theta_{\text{max}}$ & the maximum value for angle & 180 & 
		Degree\\
		$\theta_{\text{min}}$ & the minimum value for angle & 0 & 
		Degree\\
  $d_{\text{max}}$ & the maximum value for the count of joints & 9 & 
		\\
		$d_{\text{min}}$ & the minimum value for the count of joints &  3 & \\
  $S_{\text{max}}$ & the maximum value for the count of supports & 2 & 
		\\
		$S_{\text{min}}$ & the minimum value for the count of supports &  1 & 
		\\
		\bottomrule		
	\end{longtable*}
	\caption{Auxiliary Constants} \label{TblAuxConst}
\end{table}



\newpage

\bibliographystyle {plainnat}
\bibliography {refs/References.bib}




\end{document}