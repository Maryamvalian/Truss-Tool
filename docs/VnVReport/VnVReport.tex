\documentclass[12pt, titlepage]{article}

\usepackage{booktabs}
\usepackage{tabularx}
\usepackage{hyperref}
\hypersetup{
    colorlinks,
    citecolor=black,
    filecolor=black,
    linkcolor=red,
    urlcolor=blue
}
\usepackage[round]{natbib}



\begin{document}

\title{Verification and Validation Report: Truss Tool} 
\author{Maryam Valian}
\date{\today}
	
\maketitle

\pagenumbering{roman}

\section{Revision History}

\begin{tabularx}{\textwidth}{p{3cm}p{2cm}X}
\toprule {\bf Date} & {\bf Version} & {\bf Notes}\\
\midrule
15/04/2023 & 1.0 & Draft version\\
18/04/2023 & 1.1 & Update report\\
\bottomrule
\end{tabularx}

~\newpage

\section{Symbols, Abbreviations and Acronyms}

\renewcommand{\arraystretch}{1.2}
\begin{tabular}{l l} 
  \toprule		
  \textbf{symbol} & \textbf{description}\\
  \midrule 
  T & Test\\
  \bottomrule
\end{tabular}\\

\wss{symbols, abbreviations or acronyms -- you can reference the SRS tables if needed}

\newpage

\tableofcontents

\listoftables %if appropriate

\listoffigures %if appropriate

\newpage

\pagenumbering{arabic}

This document is a report on the results of a testing plan for Truss Tool.
Detailed descriptions of the tests can be found in \href{https://github.com/Maryamvalian/project741/blob/cfe06182f41c842e3b44aa0eb33d661cf8a3ce79/docs/VnVPlan/VnVPlan.pdf}{VnV Plan}

\section{Functional Requirements Evaluation}

All functional requirements have been met.

\section{Nonfunctional Requirements Evaluation}

\subsection{Reliability}
The outputs generated by Truss Tool were compared to the \href{https://valdivia.staff.jade-hs.de/fachwerk_en.html}{Existing software} with the same input data. The results were the same and the mean error between the expected value and generated value was less than 0.1.
		
\subsection{Portability}

Truss Tool runs successfully on Windows. The test was done manually.
	
\section{Comparison to Existing Implementation}	

The outputs generated by Truss Tool were compared to the \href{https://valdivia.staff.jade-hs.de/fachwerk_en.html}{Existing software}. the comparison was based on measuring mean error.

\section{Unit Testing}
The detail of the unit tests can be found in VnV Plan section. There were 12 Tests designed and all test cases are performed by test classes built with the help of Pytest. All tests succeed.

\section{Changes Due to Testing}

\section{Automated Testing}
		
\section{Trace to Requirements}
		
\section{Trace to Modules}		

\section{Code Coverage Metrics}

\bibliographystyle{plainnat}
\bibliography{../../refs/References}

\newpage{}


\end{document}