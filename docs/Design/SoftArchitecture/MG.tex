\documentclass[12pt, titlepage]{article}

\usepackage{fullpage}
\usepackage[round]{natbib}
\usepackage{multirow}
\usepackage{booktabs}
\usepackage{tabularx}
\usepackage{graphicx}
\usepackage{float}
\usepackage{hyperref}
\hypersetup{
    colorlinks,
    citecolor=blue,
    filecolor=black,
    linkcolor=red,
    urlcolor=blue
}


\newcounter{acnum}
\newcommand{\actheacnum}{AC\theacnum}
\newcommand{\acref}[1]{AC\ref{#1}}

\newcounter{ucnum}
\newcommand{\uctheucnum}{UC\theucnum}
\newcommand{\uref}[1]{UC\ref{#1}}

\newcounter{mnum}
\newcommand{\mthemnum}{M\themnum}
\newcommand{\mref}[1]{M\ref{#1}}

\begin{document}

\title{Module Guide for Truss Tool} 
\author{Maryam Valian}
\date{\today}

\maketitle

\pagenumbering{roman}

\section{Revision History}

\begin{tabularx}{\textwidth}{p{3cm}p{2cm}X}
\toprule {\bf Date} & {\bf Version} & {\bf Notes}\\
\midrule
11/03/2023 & 1.0 & Draft version\\
12/03/2023 & 1.1 & Update module decomposition\\
15/03/2023 &1.2 & Update module hierarchy\\
16/03/2023 &1.3 & Update after presentation feedback\\
23/03/2023 &1.4 & Update based on First reviewer feedback\\
30/03/2023 &1.5 & Update based on Second reviewer feedback\\
17/04/2023&1.6& Update based on Supervisor feedback\\
\bottomrule
\end{tabularx}

\newpage

\section{Reference Material}

This section records information for easy reference.

\subsection{Abbreviations and Acronyms}

\renewcommand{\arraystretch}{1.2}
\begin{tabular}{l l} 
  \toprule		
  \textbf{symbol} & \textbf{description}\\
  \midrule 
  AC & Anticipated Change\\
  DAG & Directed Acyclic Graph \\
  I & Internal Forces of a truss\\
  M & Module \\
  MG & Module Guide \\
  OS & Operating System \\
  R & Requirement\\
  SC & Scientific Computing \\
  SRS & Software Requirements Specification\\
  
  UC & Unlikely Change \\
  
  \bottomrule
\end{tabular}\\

\newpage

\tableofcontents

\listoftables

\listoffigures

\newpage

\pagenumbering{arabic}

\section{Introduction}

Decomposing a system into modules is a commonly accepted approach to developing
software.  A module is a work assignment for a programmer or programming
team~\citep{ParnasEtAl1984}.  We advocate a decomposition
based on the principle of information hiding~\citep{Parnas1972a}.  This
principle supports design for change, because the ``secrets'' that each module
hides represent likely future changes.  Design for change is valuable in SC,
where modifications are frequent, especially during initial development as the
solution space is explored.  

Our design follows the rules layed out by \citet{ParnasEtAl1984}, as follows:
\begin{itemize}
\item System details that are likely to change independently should be the
  secrets of separate modules.
\item Each data structure is implemented in only one module.
\item Any other program that requires information stored in a module's data
  structures must obtain it by calling access programs belonging to that module.
\end{itemize}

After completing the first stage of the design, the Software Requirements
Specification (SRS), the Module Guide (MG) is developed~\citep{ParnasEtAl1984}. The MG
specifies the modular structure of the system and is intended to allow both
designers and maintainers to easily identify the parts of the software.  The
potential readers of this document are as follows:

\begin{itemize}
\item New project members: This document can be a guide for a new project member
  to easily understand the overall structure and quickly find the
  relevant modules they are searching for.
\item Maintainers: The hierarchical structure of the module guide improves the
  maintainers' understanding when they need to make changes to the system. It is
  important for a maintainer to update the relevant sections of the document
  after changes have been made.
\item Designers: Once the module guide has been written, it can be used to
  check for consistency, feasibility, and flexibility. Designers can verify the
  system in various ways, such as consistency among modules, feasibility of the
  decomposition, and flexibility of the design.
\end{itemize}

The rest of the document is organized as follows. Section
\ref{SecChange} lists the anticipated and unlikely changes of the software
requirements. Section \ref{SecMH} summarizes the module decomposition that
was constructed according to the likely changes. Section \ref{SecConnection}
specifies the connections between the software requirements and the
modules. Section \ref{SecMD} gives a detailed description of the
modules. Section \ref{SecTM} includes two traceability matrices. One checks
the completeness of the design against the requirements provided in the SRS. The
other shows the relation between anticipated changes and the modules. Section
\ref{SecUse} describes the use relation between modules.

\section{Anticipated and Unlikely Changes} \label{SecChange}

This section lists possible changes to the system. According to the likeliness
of the change, the possible changes are classified into two
categories. Anticipated changes are listed in Section \ref{SecAchange}, and
unlikely changes are listed in Section \ref{SecUchange}.

\subsection{Anticipated Changes} \label{SecAchange}

Anticipated changes are the source of the information that is to be hidden
inside the modules. Ideally, changing one of the anticipated changes will only
require changing the one module that hides the associated decision. The approach
adapted here is called design for
change.

\begin{description}
\item[\refstepcounter{acnum} \actheacnum \label{acHardware}:] The specific
  hardware on which the software is running.
\item[\refstepcounter{acnum} \actheacnum \label{acInput}:] The format of the
   input data.
 \item[\refstepcounter{acnum} \actheacnum \label{acParam}:] The format of the
   input parameters.
  \item[\refstepcounter{acnum} \actheacnum \label{acVerify}:] The constraints on the input data.
\item[\refstepcounter{acnum} \actheacnum \label{acOutput}:] The format of the output data.  
\item[\refstepcounter{acnum} \actheacnum \label{acOutVeri}:] The constraints on the output data.
\item[\refstepcounter{acnum} \actheacnum \label{acDecompos}:] How to decompose forces according to input data of truss type.
\item[\refstepcounter{acnum} \actheacnum \label{acEq}:] How the equilibrium equations are defined from Input data.
\item[\refstepcounter{acnum} \actheacnum \label{acSeq}:] The implementation of the sequence data structure.
\item[\refstepcounter{acnum} \actheacnum \label{acLinalg}:] The algorithm used for the linear equation solver.
 
\end{description}
 Both AC9 and AC10 are related to libraries in Phyton. So in future, it may change. AC7 and AC8 are anticipated changes if the truss is considered to be three-dimensional. 

\subsection{Unlikely Changes} \label{SecUchange}

The module design should be as general as possible. However, a general system is
more complex. Sometimes this complexity is not necessary. Fixing some design
decisions at the system architecture stage can simplify the software design. If
these decision should later need to be changed, then many parts of the design
will potentially need to be modified. Hence, it is not intended that these
decisions will be changed.

\begin{description}
\item[\refstepcounter{ucnum} \uctheucnum \label{ucIO}:] Input/Output devices
  (Input: File and/or Keyboard, Output: File, Memory, and/or Screen).
  \item[\refstepcounter{ucnum} \uctheucnum \label{ucExternalIn}:] 
  An external source of input data should be provided for the software.
  \item[\refstepcounter{ucnum} \uctheucnum \label{ucDispaly}:] The output data are always stored in the output file.
  \item[\refstepcounter{ucnum} \uctheucnum \label{ucGoal}:] The goal of the system is to calculate support reactions and internal forces for each member.
  \item[\refstepcounter{ucnum} \uctheucnum \label{ucGoal}:] The equilibrium equations can be defined using parameters defined in the input parameter module. 

\end{description}

\section{Module Hierarchy} \label{SecMH}

This section provides an overview of the module design. Modules are summarized
in a hierarchy decomposed by secrets in Table \ref{TblMH}. The modules listed
below, which are leaves in the hierarchy tree, are the modules that will
actually be implemented.

\begin{description}
\item [\refstepcounter{mnum} \mthemnum \label{mHH}:] Hardware-Hiding Module
\item [\refstepcounter{mnum} \mthemnum \label{mIn}:] Input parameters module
\item [\refstepcounter{mnum} \mthemnum \label{mparam}:] Input verification module
\item [\refstepcounter{mnum} \mthemnum \label{mSpec}:] Specification parameters module 
\item [\refstepcounter{mnum} \mthemnum \label{mout}:] Output format module
\item [\refstepcounter{mnum} \mthemnum \label{mverify}:] Output verification module
\item [\refstepcounter{mnum} \mthemnum \label{minternal}:] Calculation module
\item [\refstepcounter{mnum} \mthemnum \label{mControl}:] Control module
\item [\refstepcounter{mnum} \mthemnum \label{msds}:] Sequence data structure module
\item [\refstepcounter{mnum} \mthemnum \label{mlinalg}:] linear equation solver module

\end{description} 
Not all of the above list will be implemented. Since M\ref{mHH} is already implemented in operating system. Also M\ref{msds} and  M\ref{mlinalg} are already implemented in Numpy and Scipy libraries of Python, respectively.

\begin{table}[h!]
\centering
\begin{tabular}{p{0.3\textwidth} p{0.6\textwidth}}
\toprule
\textbf{Level 1} & \textbf{Level 2}\\
\midrule

{Hardware-Hiding Module} & ~ \\
\midrule

\multirow{7}{0.3\textwidth}{Behaviour-Hiding Module} & Input parameters module\\
& Input verification module\\
& Specification parameters module\\
& Output format module\\
& Output verification module\\
& Calculation module.\\
& Control module\\ 

\midrule

\multirow{3}{0.3\textwidth}{Software Decision Module} & Sequence data structure module\\
& linear equation solver module\\
\bottomrule

\end{tabular}
\caption{Module Hierarchy}
\label{TblMH}
\end{table}

\section{Connection Between Requirements and Design} \label{SecConnection}

The design of the system is intended to satisfy the requirements developed in
the SRS. In this stage, the system is decomposed into modules. The connection
between requirements and modules is listed in Table~\ref{TblRT}.

\section{Module Decomposition} \label{SecMD}

Modules are decomposed according to the principle of ``information hiding''
proposed by \citet{ParnasEtAl1984}. The \emph{Secrets} field in a module
decomposition is a brief statement of the design decision hidden by the
module. The \emph{Services} field specifies \emph{what} the module will do
without documenting \emph{how} to do it. For each module, a suggestion for the
implementing software is given under the \emph{Implemented By} title. If the
entry is \emph{OS}, this means that the module is provided by the operating
system or by standard programming language libraries.  \emph{Truss Tool} means the
module will be implemented by the Truss Tool software.

Only the leaf modules in the hierarchy have to be implemented. If a dash
(\emph{--}) is shown, this means that the module is not a leaf and will not have
to be implemented.

\subsection{Hardware Hiding Modules (\mref{mHH})}

\begin{description}
\item[Secrets:]The data structure and algorithm used to implement the virtual
  hardware.
\item[Services:]Serves as a virtual hardware used by the rest of the
  system. This module provides the interface between the hardware and the
  software. So, the system can use it to display outputs or to accept inputs.
\item[Implemented By:] OS
\end{description}

\subsection{Behaviour-Hiding Modules}

\begin{description}
\item[Secrets:]The contents of the required behaviours.
\item[Services:]Includes programs that provide externally visible behaviour of
  the system as specified in the software requirements specification (SRS)
  documents. This module serves as a communication layer between the
  hardware-hiding module and the software decision module. The programs in this
  module will need to change if there are changes in the SRS.
\item[Implemented By:] Truss Tool
\end{description}

\subsubsection{Input Format Module (\mref{mIn})}

\begin{description}
\item[Secrets:]The format and structure of the input data.
\item[Services:]Converts the input data into the data structure used by the
  input parameters module.
\item[Implemented By:] [Truss Tool]
\item[Type of Module:] [Abstract Data Type]
  [Information to include for leaf modules in the decomposition by secrets tree.]
\end{description}

\subsubsection{Input verification module (\mref{mparam})}

\begin{description}
\item[Secrets:] The algorithm of the verification is isolated from the program.
\item[Services:] Checks whether the input data complies with software constraints or not. If the check fails, then an exception is thrown with an explanation of the cause.
\item[Implemented By:] [Truss Tool]
\item[Type of Module:] [ Abstract Data Type]
  [Information to include for leaf modules in the decomposition by secrets tree.]
\end{description}
\subsubsection{Specification parameters module (\mref{mSpec})}

\begin{description}
\item[Secrets:]The values of the specification parameters and constants according to SRS.
\item[Services:] Read access for the specification parameters.
\item[Implemented By:] [Truss Tool]
\item[Type of Module:] [Record]
  [Information to include for leaf modules in the decomposition by secrets tree.]
\end{description}
\subsubsection{Format output module (\mref{mout})}

\begin{description}
\item[Secrets:]The format and structure of output.
\item[Services:] Outputs results of support reaction and internal forces.
\item[Implemented By:] [Truss Tool]
\item[Type of Module:] [Abstract Data Type]
  [Information to include for leaf modules in the decomposition by secrets tree.]
\end{description}
\subsubsection{Output verification module (\mref{mverify})}

\begin{description}
\item[Secrets:]The algorithm used to check correctness of results.
\item[Services:] Verifies that the internal forces for the last joint is correct.
\item[Implemented By:] [Truss Tool]
\item[Type of Module:] [Abstract Data Type]
  [Information to include for leaf modules in the decomposition by secrets tree.]
\end{description}
\subsubsection{Calculation module (\mref{minternal})}

\begin{description}
\item[Secrets:]The algorithm used to create equilibrium equations for each joint.
\item[Services:] Calculates all equilibrium equations needed to calculate support reactions and internal forces of the truss. 
\item[Implemented By:] [Truss Tool]
\item[Type of Module:] [Abstract Data Type]
  [Information to include for leaf modules in the decomposition by secrets tree.]
\end{description}
\subsubsection{Control module (\mref{mControl})}

\begin{description}
\item[Secrets:]The algorithm for coordinating the running of the program.
\item[Services:] Provides the main program.
\item[Implemented By:] [Truss Tool]
\item[Type of Module:] [Abstract Data Type]
  [Information to include for leaf modules in the decomposition by secrets tree.]
\end{description}



\subsection{Software Decision Modules}

\begin{description}
\item[Secrets:] The design decision based on mathematical theorems, physical
  facts, or programming considerations. The secrets of this module are
  \emph{not} described in the SRS.
\item[Services:] Includes data structure and algorithms used in the system that
  do not provide direct interaction with the user. 
  % Changes in these modules are more likely to be motivated by a desire to
  % improve performance than by externally imposed changes.
\item[Implemented By:] --
\end{description}
\subsubsection{Sequence data structure module (\mref{msds})}

\begin{description}
\item[Secrets:]The data structure  for a sequence data type.
\item[Services:] Provides array creation and modifications.
\item[Implemented By:] [Numpy- Python] \url{https://numpy.org/doc/stable/reference/}
\item[Type of Module:] [Library]

 
\end{description}
\subsubsection{Linear equation solver (\mref{mlinalg})}

\begin{description}
\item[Secrets:]The algorithm used for solving a linear equation.
\item[Services:] Solves a linear equation.
\item[Implemented By:] [Scipy- Python] \url{https://docs.scipy.org/doc/scipy/reference/}
\item[Type of Module:] [Library]
  
\end{description}

\section{Traceability Matrix} \label{SecTM}

This section shows two traceability matrices: between the modules and the
requirements and between the modules and the anticipated changes.

% the table should use mref, the requirements should be named, use something
% like fref
\begin{table}[H]
\centering
\begin{tabular}{p{0.2\textwidth} p{0.6\textwidth}}
\toprule
\textbf{Req.} & \textbf{Modules}\\
\midrule
R1 & \mref{mHH},\mref{mSpec} \mref{mIn}, \mref{mparam}, \mref{mControl}\\
R2 & \mref{mHH},\mref{mout},\mref{mControl}\\
R3 & \mref{mout},\mref{mverify},\mref{minternal},\mref{mControl},\mref{msds},\mref{mlinalg}\\
R4 & \mref{mverify},\mref{mControl} \\

\bottomrule
\end{tabular}
\caption{Trace Between Requirements and Modules}
\label{TblRT}
\end{table}

\begin{table}[H]
\centering
\begin{tabular}{p{0.2\textwidth} p{0.6\textwidth}}
\toprule
\textbf{AC} & \textbf{Modules}\\
\midrule
\acref{acHardware} & \mref{mHH}\\
\acref{acInput} & \mref{mIn}\\
\acref{acParam} & \mref{mparam}\\
\acref{acVerify} & \mref{mverify}\\
\acref{acOutput} & \mref{mout}\\
\acref{acOutVeri} & \mref{mverify}\\
\acref{acDecompos} & \mref{minternal}\\
\acref{acEq} &\mref{minternal}\\
\acref{acSeq} & \mref{msds}\\
\acref{acLinalg} & \mref{mlinalg}\\
\bottomrule
\end{tabular}
\caption{Trace Between Anticipated Changes and Modules}
\label{TblACT}
\end{table}

\section{Use Hierarchy Between Modules} \label{SecUse}

In this section, the uses hierarchy between modules is
provided. \citet{Parnas1978} said of two programs A and B that A {\em uses} B if
correct execution of B may be necessary for A to complete the task described in
its specification. That is, A {\em uses} B if there exist situations in which
the correct functioning of A depends upon the availability of a correct
implementation of B.  Figure \ref{FigUH} illustrates the use relation between
the modules. It can be seen that the graph is a Directed Acyclic Graph
(DAG). Each level of the hierarchy offers a testable and usable subset of the
system, and modules in the higher level of the hierarchy are essentially simpler
because they use modules from the lower levels. Also from left to right, there is an order of using modules by Control module.

\begin{figure}[H]
\centering
\includegraphics[width=0.7\textwidth]{mg.png}
\caption{Use Hierarchy Among Modules}
\label{FigUH}
\end{figure}

%\section*{References}

\bibliographystyle {plainnat}
\bibliography {refs/References.bib}

\newpage{}

\end{document}