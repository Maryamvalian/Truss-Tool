\documentclass[12pt, titlepage]{article}

\usepackage{amsmath, mathtools}

\usepackage[round]{natbib}
\usepackage{amsfonts}
\usepackage{amssymb}
\usepackage{graphicx}
\usepackage{colortbl}
\usepackage{xr}
\usepackage{hyperref}
\usepackage{longtable}
\usepackage{xfrac}
\usepackage{tabularx}
\usepackage{float}
\usepackage{siunitx}
\usepackage{booktabs}
\usepackage{multirow}
\usepackage[section]{placeins}
\usepackage{caption}
\usepackage{fullpage}

\hypersetup{
bookmarks=true,     % show bookmarks bar?
colorlinks=true,       % false: boxed links; true: colored links
linkcolor=red,          % color of internal links (change box color with linkbordercolor)
citecolor=blue,      % color of links to bibliography
filecolor=magenta,  % color of file links
urlcolor=cyan          % color of external links
}

\usepackage{array}

\externaldocument{docs/SRS/SRS.pdf}

\begin{document}

\title{Module Interface Specification for Truss Tool}

\author{Maryam Valian}

\date{\today}

\maketitle

\pagenumbering{roman}

\section{Revision History}

\begin{tabularx}{\textwidth}{p{3cm}p{2cm}X}
\toprule {\bf Date} & {\bf Version} & {\bf Notes}\\
\midrule
16/03/2023 & 1.0 & Initial Draft\\
17/03/2023 & 1.1 & Update\\
\bottomrule
\end{tabularx}

~\newpage

\section{Symbols, Abbreviations and Acronyms}

See SRS Documentation at \href{https://github.com/Maryamvalian/project741/blob/main/docs/SRS/SRS.pdf}{here}. 
\newpage

\tableofcontents

\newpage

\pagenumbering{arabic}

\section{Introduction}

The following document details the Module Interface Specifications for
Truss Tool is software designed for engineers and students to analyze a truss.

Complementary documents include the System Requirement Specifications
and Module Guide.  The full documentation and implementation can be
found at \href{https://github.com/Maryamvalian/project741}{Truss Tool repository}.

\section{Notation}



The structure of the MIS for modules comes from \citet{HoffmanAndStrooper1995},
with the addition that template modules have been adapted from
\cite{GhezziEtAl2003}.  The mathematical notation comes from Chapter 3 of
\citet{HoffmanAndStrooper1995}.  For instance, the symbol := is used for a
multiple assignment statement and conditional rules follow the form $(c_1
\Rightarrow r_1 | c_2 \Rightarrow r_2 | ... | c_n \Rightarrow r_n )$.
Capital letters are used to indicate sequenced data type. 

The following table summarizes the primitive data types used by Truss Tool. 

\begin{center}
\renewcommand{\arraystretch}{1.2}
\noindent 
\begin{tabular}{l l p{7.5cm}} 
\toprule 
\textbf{Data Type} & \textbf{Notation} & \textbf{Description}\\ 
\midrule
character & char & a single symbol or digit\\
integer & $\mathbb{Z}$ & a number without a fractional component in (-$\infty$, $\infty$) \\
natural number & $\mathbb{N}$ & a number without a fractional component in [1, $\infty$) \\
real & $\mathbb{R}$ & any number in (-$\infty$, $\infty$)\\
\bottomrule
\end{tabular} 
\end{center}

\noindent
The specification of  Truss Tool uses some derived data types: sequences, strings, and
tuples. Sequences are lists filled with elements of the same data type. Strings
are sequences of characters. Tuples contain a list of values, potentially of
different types. In addition, Truss Tool uses functions, which
are defined by the data types of their inputs and outputs. Local functions are
described by giving their type signature followed by their specification.

\section{Module Decomposition}

The following table is taken directly from the Module Guide document for this project.

\begin{table}[h!]
\centering
\begin{tabular}{p{0.3\textwidth} p{0.6\textwidth}}
\toprule
\textbf{Level 1} & \textbf{Level 2}\\
\midrule

{Hardware-Hiding Module} & ~ \\
\midrule

\multirow{7}{0.3\textwidth}{Behaviour-Hiding Module} & Input parameters module\\
& Input verification module\\
& Specification parameters module\\
& Output format module\\
& Output verification module\\
& Equilibrium equations for free body of truss module. \\
& Force decomposing module\\
& Equilibrium equations for all joints module.\\
& Control module\\ 

\midrule

\multirow{3}{0.3\textwidth}{Software Decision Module} & Sequence data structure module\\
& linear equation solver module\\
\bottomrule

\end{tabular}
\caption{Module Hierarchy}
\label{TblMH}
\end{table}
\newpage
~\newpage

\section{MIS of Control module \label{mControl} }

\subsection{Module}

Main

\subsection{Uses}
Input Module, Reaction Module, Internal Forces Module, Output Verification Module

\subsection{Syntax}

\subsubsection{Exported Constants}
Not applicable.

\subsubsection{Exported Access Programs}

\begin{center}
\begin{tabular}{p{2cm} p{4cm} p{4cm} p{2cm}}
\hline
\textbf{Name} & \textbf{In} & \textbf{Out} & \textbf{Exceptions} \\
\hline
Main & - & - & - \\
\hline
\end{tabular}
\end{center}

\subsection{Semantics}

\subsubsection{State Variables}
None.

\subsubsection{Environment Variables}

This module has external interaction with an input file, an output file.

\subsubsection{Assumptions}

The path and the name of the given input file are correct.

\subsubsection{Access Routine Semantics}

\noindent Main():  
\begin{itemize}
\item transition: Modifies the state of the Input module and the environment variables for the Output modules by following these steps:
\item output: 
\item exception:  Input file not found
\end{itemize}

\subsubsection{Local Functions}

None.

\section{MIS of Input Module \label{mInput} }

\subsection{Module}

Inputs 

\subsection{Uses}
Input Verification Module

\subsection{Syntax}
\begin{tabular}{p{3cm} p{1cm} p{1cm} >{\raggedright\arraybackslash}p{9cm}}
\toprule
\textbf{Name} & \textbf{In} & \textbf{Out} & \textbf{Exceptions} \\
\midrule
load\_params & string & - &  Input File Error \\
verify\_params & - & - & Input Parameters Error\\
$n$ & -& $\mathbb{N}$\\
$m$ & -& $\mathbb{N}$\\
$J_n$ & -& $\mathbb{R}$\\
$M_m$ & - & $\mathbb{N}$\\
$F_m$ & - & $\mathbb{R}$\\
$sp$ & - & $\mathbb{N}$\\
$sr$ & - & $\mathbb{N}$\\
\bottomrule
\end{tabular}

\subsubsection{Exported Constants}
Not applicable.

\subsubsection{Exported Access Programs}

\begin{center}
\begin{tabular}{p{2cm} p{4cm} p{4cm} p{2cm}}
\hline
\textbf{Name} & \textbf{In} & \textbf{Out} & \textbf{Exceptions} \\
\hline
Inputs & input plain text file & Parameters & FileError \\
\hline
\end{tabular}
\end{center}

\subsection{Semantics}
\subsubsection{State Variables}
$\#$ from R1 and R2:\\
$n$ : $\mathbb{N}$ \\
$m$ : $\mathbb{N}$ \\
$J_n$ : $\mathbb{R}$ \\
$M_m$ : $\mathbb{N}$ \\
$F_m$ : $\mathbb{R}$ \\
$sp$ : $\mathbb{N}$ \\
$sr$ : $\mathbb{N}$ \\
\newline
$\#$ from R3:\\
$px$ : $\mathbb{R}$ \\
$py$ : $\mathbb{R}$ \\
$ry$ : $\mathbb{R}$ \\
\newline
$\#$ from R4:\\
$I_m$ : $\mathbb{R}$ \\



\subsubsection{Environment Variables}

InputFile: sequence of strings in the text file.

\subsubsection{Assumptions}

load_params will be called before the values of any state variables will be accessed.

The file contains the string equivalents of the numeric values for each input parameter
in order, each on a new line.



\subsubsection{Access Routine Semantics}

\noindent Main():  
\begin{itemize}
\item transition: Modifies the state of the Input module and the environment variables for the Output modules by following these steps:
\item output: 
\item exception:  Input file not found
\end{itemize}

\subsubsection{Local Functions}

None.

\newpage

\bibliographystyle {plainnat}
\bibliography {refs/References.bib}

\newpage

\section{Appendix} \label{Appendix}

Not applicable.

\end{document}