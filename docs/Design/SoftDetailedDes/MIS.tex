\documentclass[12pt, titlepage]{article}

\usepackage{amsmath, mathtools}

\usepackage[round]{natbib}
\usepackage{amsfonts}
\usepackage{amssymb}
\usepackage{graphicx}
\usepackage{colortbl}
\usepackage{xr}
\usepackage{hyperref}
\usepackage{longtable}
\usepackage{xfrac}
\usepackage{tabularx}
\usepackage{float}
\usepackage{siunitx}
\usepackage{booktabs}
\usepackage{multirow}
\usepackage[section]{placeins}
\usepackage{caption}
\usepackage{fullpage}

\hypersetup{
bookmarks=true,     % show bookmarks bar?
colorlinks=true,       % false: boxed links; true: colored links
linkcolor=red,          % color of internal links (change box color with linkbordercolor)
citecolor=blue,      % color of links to bibliography
filecolor=magenta,  % color of file links
urlcolor=cyan          % color of external links
}

\usepackage{array}

\externaldocument{docs/SRS/SRS.pdf}

\begin{document}

\title{Module Interface Specification for Truss Tool}

\author{Maryam Valian}

\date{\today}

\maketitle

\pagenumbering{roman}

\section{Revision History}

\begin{tabularx}{\textwidth}{p{3cm}p{2cm}X}
\toprule {\bf Date} & {\bf Version} & {\bf Notes}\\
\midrule
16/03/2023 & 1.0 & Initial Draft\\
17/03/2023 & 1.1 & Update Control module\\
19/03/2023 & 1.2 & Update formulas\\
20/03/2023 & 1.3 & Update other modules\\
23/03/2023 & 1.4 & Update based on first Reviewer Feedback\\
30/03/2023 & 1.5 & Update based on second Reviewer Feedback\\
18/04/2023&1.6& Update based on Supervisor feedback\\
\bottomrule
\end{tabularx}

~\newpage

\section{Symbols, Abbreviations and Acronyms}

See SRS Documentation at \href{https://github.com/Maryamvalian/project741/blob/main/docs/SRS/SRS.pdf}{here}. 
\newpage

\tableofcontents

\newpage

\pagenumbering{arabic}

\section{Introduction}

The following document details the Module Interface Specifications for
Truss Tool is software designed for engineers and students to analyze a truss.

Complementary documents include the System Requirement Specifications
and Module Guide.  The full documentation and implementation can be
found at \href{https://github.com/Maryamvalian/project741}{Truss Tool repository}.

\section{Notation}



The structure of the MIS for modules comes from \citet{HoffmanAndStrooper1995},
with the addition that template modules have been adapted from
\cite{GhezziEtAl2003}.  The mathematical notation comes from Chapter 3 of
\citet{HoffmanAndStrooper1995}.  For instance, the symbol := is used for a
multiple assignment statement and conditional rules follow the form $(c_1
\Rightarrow r_1 | c_2 \Rightarrow r_2 | ... | c_n \Rightarrow r_n )$.
Capital letters are used to indicate sequenced data type. 

The following table summarizes the primitive data types used by Truss Tool. 

\begin{center}
\renewcommand{\arraystretch}{1.2}
\noindent 
\begin{tabular}{l l p{7.5cm}} 
\toprule 
\textbf{Data Type} & \textbf{Notation} & \textbf{Description}\\ 
\midrule
character & char & a single symbol or digit\\
integer & $\mathbb{Z}$ & a number without a fractional component in (-$\infty$, $\infty$) \\
natural number & $\mathbb{N}$ & a number without a fractional component in [1, $\infty$) \\
real & $\mathbb{R}$ & any number in (-$\infty$, $\infty$)\\
\bottomrule
\end{tabular} 
\end{center}

\noindent
The specification of  Truss Tool uses some derived data types: sequences, strings, and
tuples. Sequences are lists filled with elements of the same data type. Strings
are sequences of characters. Tuples contain a list of values, potentially of
different types. In addition, Truss Tool uses functions, which
are defined by the data types of their inputs and outputs. Local functions are
described by giving their type signature followed by their specification.

\section{Module Decomposition}

The following table is taken directly from the Module Guide document for this project.

\begin{table}[h!]
\centering
\begin{tabular}{p{0.3\textwidth} p{0.6\textwidth}}
\toprule
\textbf{Level 1} & \textbf{Level 2}\\
\midrule

{Hardware-Hiding Module} & ~ \\
\midrule

\multirow{7}{0.3\textwidth}{Behaviour-Hiding Module} & Input parameters module\\
& Input verification module\\
& Specification parameters module\\
& Output format module\\
& Output verification module\\
& Support reactions module. \\
& Force decomposing module\\
& Internal force module\\
& Control module\\ 

\midrule

\multirow{3}{0.3\textwidth}{Software Decision Module} & Sequence data structure module\\
& linear equation solver module\\
\bottomrule

\end{tabular}
\caption{Module Hierarchy}
\label{TblMH}
\end{table}
\newpage
~\newpage

\section{MIS of Control module \label{mControl} }

\subsection{Module}

Main

\subsection{Uses}
Inputs Module, Internals Module, Outputs Module

\subsection{Syntax}

\subsubsection{Exported Constants}
Not applicable.

\subsubsection{Exported Access Programs}

\begin{center}
\begin{tabular}{p{2cm} p{4cm} p{4cm} p{2cm}}
\hline
\textbf{Name} & \textbf{In} & \textbf{Out} & \textbf{Exceptions} \\
\hline
Main & - & - & - \\
\hline
\end{tabular}
\end{center}

\subsection{Semantics}

\subsubsection{State Variables}

$\#$ from R3:\\
$px$ : $\mathbb{R}$ \\
$py$ : $\mathbb{R}$ \\
$ry$ : $\mathbb{R}$ \\
\newline
$\#$ from R4:\\
$I_m$ : $\mathbb{R}$ \\

\subsubsection{Environment Variables}

Not Applicable.

\subsubsection{Assumptions}

Not Applicable.

\subsubsection{Access Routine Semantics}

\noindent Main():  
\begin{itemize}
\item transition: Modifies the state of the Input module and the environment variables for the Output modules by following steps:


$(n,m,J,M,F,S_p,S_r)=Inputs(FileName)$

$(P_x,P_y,R_y,I)=Internals(n,m,J,M,F,S_p,S_r)$

$Outputs(P_x,P_y,R_y,I)$



\end{itemize}

\subsubsection{Local Functions}

None.
%----------------------------------------------------------------
\section{MIS of Input Module \label{mInput} }

\subsection{Module}

Inputs 

\subsection{Uses}
Input Verification Module

\subsection{Syntax}
\begin{tabular}{p{3cm} p{1cm} p{1cm} >{\raggedright\arraybackslash}p{9cm}}
\toprule
\textbf{Name} & \textbf{In} & \textbf{Out} & \textbf{Exceptions} \\
\midrule
load\_params & string & - &  Input File Error \\
verify\_params & - & - & Input Parameters Error\\
$n$ & -& $\mathbb{N}$ \\
$m$ & -& $\mathbb{N}$\\
$J[j ]\,$ & -& $\mathbb{R}$\\
$M_m$ & - & $\mathbb{N}$\\
$F_m$ & - & $\mathbb{R}$\\
$sp$ & - & $\mathbb{N}$\\
$sr$ & - & $\mathbb{N}$\\
\bottomrule
\end{tabular}

\subsubsection{Exported Constants}
Not applicable.

\subsubsection{Exported Access Programs}

\begin{center}
\begin{tabular}{p{2cm} p{4cm} p{4cm} p{2cm}}
\hline
\textbf{Name} & \textbf{In} & \textbf{Out} & \textbf{Exceptions} \\
\hline
Inputs & input plain text file & Parameters & FileError \\
\hline
\end{tabular}
\end{center}

\subsection{Semantics}
\subsubsection{State Variables}
$\#$ from R1 and R2:\\
$n$ : Integer \\
$m$ : Integer \\
$J[\,j\,]$ : Sequence of pair of Real numbers (x,y) \\
$M[\,i\,]$ : Sequence of pair of Integers (end1, end2)\\
$F[\,j\,]$ : Sequence of pair of Real numbers  \\
$S_p$ : Integer \\
$S_r$ : Integer \\




\subsubsection{Environment Variables}

InputFile: sequence of strings in the text file.

\subsubsection{Assumptions}
\begin{itemize}
    \item {Load parameters will be called before the values of any state variables will be accessed.}
    \item{The file contains the string equivalents of the numeric values for each input parameter. The order is important.}
\end{itemize}
\subsubsection{Access Routine Semantics}

\noindent Load-Parameter():  
\begin{itemize}
\item transition: Modifies the state variables from the input file:
\begin{itemize}
    \item Read sequentially from file and loads state variables from R1 and R2.
    \item verify-input()
\end{itemize}
\item output: None 
\item exception:  file-Error, input-type-mismatch-Error
\end{itemize}

\noindent Verify-Input():
\begin{itemize}
\item transition: Modifies the state variables from the input file:
\begin{itemize}
    \item Read constants from specification module.
    \item Check constraints given in Table \ref{TblSpecParams}.
    \begin{table}[!h]
\caption{Constraint check table} \label{TblSpecParams}
\renewcommand{\arraystretch}{1.2}
\noindent \begin{longtable*}{l l} 
  \toprule
  \textbf{Constraint} & \textbf{Error} \\
  \midrule 
  $n_{min}$ \leq $n$ \leq $n_{max} $ & InvalidJonits \\
  $m$ \leq $m_{max} $ & InvalidMembers\\
  $m < n$ & InvalidMembers\\
  $J_{min} \leq J[\,j\,] \leq J_{max} $ & InvalidLocation \\
  $F_{min}$ \leq $F[\,j\,] $ \leq $F_{max} $ & InvalidForce \\
  $Sp>n,Sp<0$ & InvalidIndex\\
  $Sr>n, Sr<0$ & InvalidIndex\\
  $Sp=Sr=0$& NoSupportDefined\\
  \bottomrule
\end{longtable*}
\end{table}
\end{itemize}
\item output: None 
\item exception:  Bad-parameters-Error
\end{itemize}
%---------------------------------------------------------
\section{MIS of Specification Parameters \label{mSpec} }

\subsection{Module}

Specification

\subsection{Uses}
None.

\subsection{Syntax}

\subsubsection{Exported Constants}
$\#$ From Table.2 SRS: \\
$n_{min}:=3$\\
$n_{max}:=20$\\
$m_{max}:=30$\\
$J_{max}:=200$\\
$J_{min}:=0$\\
$F_{max}:=100000$\\
$F_{min}:=-100000$\\
$error_{value}:=0.1$



\subsubsection{Exported Access Programs}

\subsection{Semantics}
\subsubsection{State Variables}
None.

\subsubsection{Environment Variables}
None.
%***************************************************************

\section{MIS of Internals Module\label{mSpec} }

\subsection{Module}

Internals

\subsection{Uses}

Sequence data structure module\\
linear equation solver module\\



\subsection{Syntax}

\subsubsection{Exported Constants}
None.
\subsubsection{Exported Access Programs}
\begin{center}
\begin{tabular}{p{2cm} p{4cm} p{4cm} p{2cm}}
\hline
\textbf{Name} & \textbf{In} & \textbf{Out} & \textbf{Exceptions} \\
\hline
React & $n,m,J,F,S_p,S_r$  & $P_x,P_y,R_y$& ReactFailed \\
Internals & $n,m,J,M,F,S_p,S_r$  &$I[\,i\,]$ & InternalFailed \\
\hline
\end{tabular}
\end{center}
Exception ReactFailed will show to the user if a correct member is not defined by the user so the angle between the member and the x-axis cannot be calculated. Exception InternalFailed will show to the user if the equations cannot be solved.
\subsection{Semantics}
\subsubsection{State Variables}
None.

\subsubsection{Environment Variables}
None.

\subsubsection{Assumptions}
None.
\subsubsection{Access Routine Semantics}
\begin{itemize}
    \item Calculates: $Out:=P_x,P_y,R_y,I\,[i\,]$ by following steps:\\
    \#\textit{Find Support reactions $(P_x,P_y,R_y)$}\\
$P_x=-\sum F_x$\\
\[A=
   \left[ {\begin{array}{cc}
    1 & 1 \\
    J\,[S_p\,] & J[\,S_r\,] \\
  \end{array} } \right]\]
\[B=
   \left[ {\begin{array}{cc}
    -\sum F[\,i\,] \\
    -\sum (J[\,i\,]* F[\,i\,])  \\
  \end{array} } \right]
\]\\ 
$A*[\,P_y,R_y\,] =B$

$(P_y,R_y) = Solve(A,B)$\\

\#\textit{Find Internal Forces $I_m$}\\

$\theta[\,i\,]=Tan^-1(J[\,M[i,end1\,]],J[M[i,end2]])$

\#\textit{Compute $a[\,i,j\,] Elements of A,and compute b[j]$ elements of $B$} from equilibrium equations where $AI=B$:\\

Build Equilibrium equations in the X-axis direction:
$a[i,j]=\cos\theta[i]$

$b_j=\sum \textit{All forces and reactions at j}$\\

Or Build Equilibrium equations in the Y-axis direction:\\

$a[i,j]=\sin\theta[i]$

$b_j=\sum \textit{All forces and reactions at j}$\\

$I=Solve(A,B)$


    
    
\end{itemize}
%----------------------------------------------------------------
\subsection{MIS of Output Verification Module \label{mSpec} }

\subsection{Module}

verify-out

\subsection{Uses}
None.

\subsection{Syntax}

\subsubsection{Exported Constants}
None.
\subsubsection{Exported Access Programs}
\begin{center}
\begin{tabular}{p{2cm} p{4cm} p{4cm} p{2cm}}
\hline
\textbf{Name} & \textbf{In} & \textbf{Out} & \textbf{Exceptions} \\
\hline

Verify-0ut & $check(I_m,J_j):\mathbb{R}*\mathbb{R} \rightarrow {Boolean}$  & Valid/NotValid & verifyFailed \\
\hline
\end{tabular}
\end{center}
\subsection{Semantics}
\subsubsection{State Variables}
None.

\subsubsection{Environment Variables}
None.

\subsubsection{Assumptions}
None.
\subsubsection{Access Routine Semantics}
\begin{itemize}
    \item Exception: Exep:=($\sum { (I_m(n) + F_m(n), P_x(n),P_y(n),R_y(n))} \neq 0)$ \Rightarrow VerifyFailed
\end{itemize}
\subsubsection{Assumptions}
None.

%-------------------------------------------------------------------
\subsection{MIS of Output Module \label{mSpec} }

\subsection{Module}

Output

\subsection{Uses}
None.

\subsection{Syntax}

\subsubsection{Exported Constants}

output file name.
\subsubsection{Exported Access Programs}
None.
\subsection{Semantics}
\subsubsection{State Variables}
None.

\subsubsection{Environment Variables}
output file

\subsubsection{Assumptions}
None.

\subsection{}{Assumptions}
None.
\subsubsection{Access Routine Semantics}
\begin{itemize}
    \item transition: Write to file the following: the input 
parameters from Param, and the calculated values $I_m, P_x,P_y,R_y$. 

\end{itemize}


\newpage

\bibliographystyle {plainnat}
\bibliography {refs/References.bib}

\newpage

\section{Appendix} \label{Appendix}

Not applicable.

\end{document}