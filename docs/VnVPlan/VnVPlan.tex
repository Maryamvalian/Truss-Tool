\documentclass[12pt, titlepage]{article}

\usepackage{booktabs}
\usepackage{tabularx}
\usepackage{hyperref}
\hypersetup{
    colorlinks,
    citecolor=blue,
    filecolor=black,
    linkcolor=red,
    urlcolor=blue
}
\usepackage[round]{natbib}

\begin{document}

\title{ System Verification and Validation Plan for Truss Tool} 
\author{Maryam Valian}
\date{\today}
	
\maketitle

\pagenumbering{roman}

\section{Revision History}

\begin{tabularx}{\textwidth}{p{3cm}p{2cm}X}
\toprule {\bf Date} & {\bf Version} & {\bf Notes}\\
\midrule
13-02-2023 & 1.0 & First Draft of VnV plan\\
14-02-2023 & 1.1& Updating Test cases   \\
16-02-2023 & 1.2& Updating Tests for NFRs   \\
25-02-2023 & 1.3& Updating According to first reviewer feedback   \\
26-02-2023 & 1.4& Updating According to second reviewer feedback   \\
27-03-2023 & 1.5& Updating based on Professor's feedback \\
17-04-2023 & 1.6& Updating Unit Test after design \\



\bottomrule
\end{tabularx}

\newpage

\tableofcontents

\newpage

\section{Symbols, Abbreviations and Acronyms}

\renewcommand{\arraystretch}{1.2}
\begin{tabular}{l l} 
  \toprule		
  \textbf{symbol} & \textbf{description}\\
  \midrule 
  T & Test\\
  SRS & Software Requirements Specification\\
  R & Functional Requirements\\
  NFR & Nonfunctional Requirements\\
  N/A & Not Applicable\\
  VnV & Verification and Validation\\
  
\bottomrule
\end{tabular}\\


\newpage
This document provides an overview of the Verification and Validation
 plan for Truss Tool. The general information is introduced in section 3. The verification plans and the test descriptions are explained in sections 4,5, and 6, respectively.
\pagenumbering{arabic}
\section{General Information}


\subsection{Summary}

In this document, we explain how Truss Tool is being tested. Truss Tool is software for truss analysis. Users can define a specific truss, by giving the exact location of the joints, members, supports and external forces.  Truss Tool calculates the reaction of the supports and internal forces. 

\subsection{Objectives}

Our objective is to build confidence in the software's correctness and increase its reliability. The functional requirements will be tested according to section 5.1, and Non-functional requirements will be tested according to section 5.2. We do not measure usability for this project. In this phase, we did not develop a graphical visualization for the truss. The usability will be considered when a graphical module is added to the project.

\subsection{Relevant Documentation}
Relevant documentation can be found by providing direct links. The VnV plan is highly related to the VnV report and we will report the results of the designed test in that document. Please see the Problem statement for goal definition and the SRS document for requirement analysis.
\begin{itemize}
	\item 
	\href{https://github.com/Maryamvalian/project741/blob/main/docs/ProblemStatementAndGoals/ProblemStatement.pdf}{Problem
	 Statement}
	\item 
	\href{https://github.com/Maryamvalian/project741/blob/main/docs/SRS/SRS.pdf}{SRS} 
	\item \href{https://github.com/Maryamvalian/project741/blob/main/docs/VnVReport/VnVReport.pdf}{VnV
	 report}
\end{itemize}

\section{Plan}
In this section, we explain the VnV plan of Truss Tool. The first section will introduce the VnV team members and their roles in the project's verification. Next, the verification plans of the SRS, design, the VnV verification plan, implementation, automated testing, and software validation plan are given. For static analysis, we use Pylint library of Python as a code inspection. 

\subsection{Verification and Validation Team} \label{VTeam}
The VnV team members and their roles are as follows:
\begin{itemize}
\item {Maryam Valian reviews the whole project as the author.}
\item {Dr.\  Smith reviews the whole project as supervisor.}
\item{Lesley Wheat reviews the whole project as the domain expert.}
\item{Joachim de Fourestier reviews the VnV plan as the secondary reviewer.}
\item{Jason Balaci reviews the MIS as the secondary reviewer.}
\item{Zehong Wong failed to review the SRS as the secondary reviewer.}

\end{itemize}

\subsection{SRS Verification Plan} \label{SRSvplan}
The SRS will be reviewed by  Smith and Lesely Wheat. Reviewers can give feedback and revision suggestions to the author by creating issues on GitHub. It is the author's responsibility to check the submitted issues regularly and make necessary modifications.

\subsection{Design Verification Plan} \label{Sec_Design_verific}

The design document MIS will be reviewed by Dr.\  Smith, Lesley Wheat and Jason Balaci. Reviewers can give feedback and revision suggestions to the author by creating issues on GitHub. It is the author's responsibility to check the submitted issues regularly and make necessary modifications.

\subsection{Verification and Validation Plan Verification Plan} \label{Sec_VofVnV}
The VnV plan document will be reviewed by Dr.\  Smith, Lesley Wheat and Joachim de Fourestier. Reviewers can give feedback and revision suggestions to the author by creating issues on GitHub. It is the author's responsibility to check the submitted issues regularly and make necessary modifications.
\subsection{Implementation Verification Plan} \label{Sec_implementation}
The implementation of the software will be verified using several techniques involving manual or automated interactions. The software will be developed in Python programming language. The code evaluation will involve unit and system testing. The tools for the
evaluations will be mentioned in section \ref{sec_automat}. Furthermore, the test cases used in system and unit testing will be stated in sections \ref{sec_systest} and \ref{sec_unitest}, respectively.

\subsection{Automated Testing and Verification Tools} \label{sec_automat}
Truss Tool will be tested and verified by Pytest as an automated tool for unit testing and system testing. Automated testing will be implemented for individual units as well as the integrated system.

\subsection{Software Validation Plan}
 Validation is the process of comparing the outputs of models to experimental values. The validation of the Truss Tool cannot be measured because we don't have real experimental data. 
\section{System Test Description} \label{sec_systest}
	
\subsection{Tests for Functional Requirements}
The functional requirements are described in the \href{https://github.com/Maryamvalian/project741/blob/main/docs/SRS/SRS.pdf}{SRS}. Truss Tool will detect invalid inputs and assure that the calculated outputs are correct. Testing R1 and R2 will be explained in the input verification section. R3 and R4 will be explained in the output verification test. 
\subsubsection{Input Verification}
The inputs will be tested to satisfy R1 and R2 from Truss Tool SRS. Specifically, this test will ensure values of the inputs align with the input constraints. Table \ref{tbl_tc} displays the inputs and outputs of test cases for the input constraints tests.
\begin{table}
 \begin{tabular}{|c|c|c|c|} 
 \hline
TC &$N$ &$J$,$M$,$F,S_r,S_p$& \textbf{Output}\\ 
 \hline
 Valid & 3& $J=\{(0,0),(1,1),(2,0)\}$& (0,25,25) \\ 
 &&$M=\{(1,2),(2,3),(1,3)\}$&(-35,-35,25)\\ 
&&$F=\{(0,0),(0,-50),(0,0)\}$&\\ 
 &&$S_p=1,S_r=3$&\\
 \hline
Invalid Force & 3& $J=\{(0,0),(1,1),(2,0)\}$& Invalid force  \\ 
 &&$M=\{(1,2),(2,3),(1,3)\}$&Exception\\ 
&&$F=\{(0,0),(0,100001),(0,0)\}$&\\ 
 &&$S_p=1,S_r=3$&\\
  \hline
Invalid Joint & 2& $J=\{(0,0),(1,1)\}$& Invalid Joint  \\ 
 &&$M=\{(1,2)\}$&Exception\\ 
&&$F=\{(0,0),(0,-50),(0,0)\}$&\\ 
 &&$S_p=1,S_r=0$&\\
   \hline
Invalid Location & 3& $J=\{(0,0),(1,201),(1,0)\}$& Invalid Location \\ 
 &&$M=\{(1,2),(1,3),(2,3)\}$&Exception \\ 
&&$F=\{(0,0),(0,-50),(0,0)\}$&\\ 
 &&$S_p=1,S_r=3$&\\
    \hline
Invalid Connection & 3& $J=\{(0,0),(1,1),(2,0)\}$& Invalid Connection  \\ 
 &&$M=\{(1,1),(3,3),(2,2)\}$&Exception\\ 
&&$F=\{(0,0),(0,-50),(0,0)\}$&\\ 
 &&$S_p=1,S_r=3$&\\
    \hline
Invalid Members & 4& $J=\{(0,0),(1,1),(2,0),(3,1)\}$& Invalid Members  \\ 
 &&$M=\{(1,2),(1,3),(2,3)\}$&Exception\\ 
&&$F=\{(0,0),(0,-50),(0,0)\}$&\\ 
 &&$S_p=1,S_r=3$&\\
    \hline
Invalid Index & 3& $J=\{(0,0),(1,1),(2,0)\}$& Invalid 
 Index  \\ 
 &&$M=\{(1,2),(1,3),(2,3)\}$&Exception\\ 
&&$F=\{(0,0),(0,-50),(0,0)\}$&\\ 
 &&$S_p=4,S_r=5$&\\
    \hline
Invalid Supports & 3& $J=\{(0,0),(1,201),(1,0)\}$& No Support  \\ 
 &&$M=\{(1,2),(1,3),(2,3)\}$&Exception\\ 
&&$F=\{(0,0),(0,-50),(0,0)\}$&\\ 
 &&$S_p=0,S_r=0$&\\
 
 \hline
 
 
 
\end{tabular}
\caption{\label{tbl_tc}Test cases For input Verification.}
\end{table}		
\paragraph{Input Verification test}

\begin{enumerate}

\item{T-1: Expected Input\\}

Control: Automatic
					
Initial State: N/A
					
Input: Valid
					
Output: Support Reactions (25,25), Internal forces (-35,-35,25)

Test Case Derivation: Derived from existing software 2D-Truss Analysis.
					
How the test will be performed: It will be performed by test classes built with the help of Pytest
					
\item{T-2: Invalid Number of joints \\}

Control: Automatic
					
Initial State: N/A
					
Input: Invalid Joint
					
Output: Exception: Invalid Joints

Test Case Derivation: N input is out of bound for 20 and 2.

How the test will be performed: It will be performed by test classes built with the help of Pytest.
\item{T-3: Invalid amount of external force \\}

Control: Automatic
					
Initial State: N/A
					
Input: Invalid Force
					
Output: Exception: Invalid Force

Test Case Derivation: Amount of the external force/forces out of bound. 

How the test will be performed: It will be performed by test classes built with the help of Pytest.
\item{T-4: Invalid index of joints \\}

Control: Automatic
					
Initial State: N/A
					
Input: Invalid Index
					
Output: Exception: Invalid Index

Test Case Derivation: Any reference to joints must be in the range of 1 to N. The support Index can be zero in case of no existing support, but both indices of support cannot be zero at the same time.

How the test will be performed: It will be performed by test classes built with the help of Pytest.


\item{T-5: No support defined \\}

Control: Automatic
					
Initial State: N/A
					
Input: Invalid Supports
					
Output: Exception: Invalid Supports

Test Case Derivation: There must be at least one support, otherwise the truss moves. 

How the test will be performed: It will be performed by test classes built with the help of Pytest.
\item{T-6: Invalid Connection \\}

Control: Automatic
					
Initial State: N/A
					
Input: Invalid Connection
					
Output: Exception: Invalid Connection

Test Case Derivation: A valid member connects to different joints.

How the test will be performed: It will be performed by test classes built with the help of Pytest.

\item{T-7: Invalid Members \\}

Control: Automatic
					
Initial State: N/A
					
Input: Invalid Members
					
Output: Exception: Invalid Members

Test Case Derivation: Member numbers cannot be less than the Joints number.

How the test will be performed: It will be performed by test classes built with the help of Pytest.

\end{enumerate}

\subsubsection{Output verification}
\begin{table} [h!]

 \begin{tabular}{|c|c |c |c|} 
 \hline
 \textbf{Test cases}&N& \textbf{input}& Results\\ 
 \hline
 Valid  & $3$&$J=\{(0,0),((1,1),(2,0)\}$&$Px=0,Py=25$\\
 N3m3&  &  $M=\{(1,2),((1,3),(2,3)\}$&$Ry=25$\\
 &  & $F=\{(0,0),(0,-50),(0,0)\}$&$I=-35,-35,25$\\
 & & $P=2,R=3$&\\
 \hline
 Valid  & $4$&$J=\{(0,0),((1,1),(1,0),(2,0)\}$&$Px=0,Py=0$\\
 N4m5&  &$M=\{(1,2),((1,3),(2,3)(2,4),(3,4)$\}$& $Ry=-10$\\
 &  & $F=\{(0,0),(0,0)(0,10)(0,0)\}$&$I=0,0,-10,0,0$\\
 & & $P=1,R=2$&\\
 \hline
  Valid  & $5$&$J=\{(0,0),((1,1),(2,0),(3,1),(4,0)\}$&$Px=0,Py=37.5$\\
 N5m7&  &  $M=\{(1,2),((1,3),(2,3)$&$Ry=12.5$\\
 &  & $,(2,4),(3,4),(3,5)\}$&$\\
 &  &$F=\{(0,0),(0,-50),$ &$I=-53.0,37.5,$\\
 &  & $(0,0),(0,0)(0,0)\}$&$-17.6,-25,17.6,$\\
 & & $P=1,R=5$&$12.5,-17.6$\\
 \hline
  Valid  & $10$&$J=\{(0,0),((2,0),(0,2),(2,2),$&$Px=0,Py=300$\\
 Crane&  & $(0,4),(2,4)(0,6),(2,6),(2,8),(6,6)\}$&$Ry=-200$\\
 &  & $M=\{(1,2),(1,3),(2,3),(2,4),$&\\
 & & $(3,4),(3,5),(4,5),(4,6),(5,6),$ &\\
 & & $(5,7),(6,7),(6,8),(7,8),(8,10),$ &$I=0,200,0,-300,0,200,$\\
 & & $(7,9),(8,9),(9,10)\}$ &$0,-300,0,200,0,$\\
 &  &$F=\{(0,0),(0,0),(0,0),(0,0),(0,0),$&$-300,-200,-200,282.8,$\\
 &  & $(0,0),(0,0)(0,0),(0,0),(0,-100)\}$&$-300,223.6$\\
 & & $P=2,R=1$&\\
 \hline
\end{tabular}

\caption{\label{tbl_tcR}Test cases for output verification Pseudo-oracle test}

\end{table}	
The output will be tested as follows:
\paragraph{Output Verification test}

\begin{enumerate}

\item{T-8: Zero summation for the last joint\\}

Control: Automatic
					
Initial State: All internal forces are already calculated.
					
Input: Internal forces array $I$\\
					
Output: Exception: Output Error
					
How the test will be performed: It will be performed by Truss Tool. The summation of internal forces in the last joint will be less than a small number $e$. Otherwise, it will generate an exception message.
\item{T-9: Pseudo-oracle test\\}
Control: Automatic
					
Initial State: All internal forces are already calculated.
					
Input: Valid n4m5 \\
					
Output:  The mean error between results.
					
How the test will be performed: It will be performed by the help of Pytest.

\end{enumerate}

\subsection{Tests for Nonfunctional Requirements}
This section will define the tests to ensure Truss Tool satisfies the nonfunctional requirements seen in the SRS document of Truss Tool. 
\subsubsection{Reliability}


		
\paragraph{Performance measurement}

\begin{enumerate}

\item{T-9: Reliability\\}

Type: Automatic
					
Initial State: Results of Truss tool on Test cases are ready.					
Input/Condition: The user has entered the inputs correctly.

					
Output/Result: The mean error between the results of truss tool and pseudo-oracle.
					
How the test will be performed: The results from table \ref{tbl_tcR}  will be compared to the results of Truss Tool and automatically checked that the difference is less than a small number.

	

\end{enumerate}

%\subsubsection{Area of Testing2}
\subsection{Traceability Between Test Cases and Requirements}
The purpose of the traceability matrices is to provide easy references on what has to be additionally modified if a certain component is changed. Every time a component is changed, the items in the column of that component that is marked with an X may have to be modified as well. Table \ref{tbl_tr_TC} shows which test cases are supporting which requirements.


\begin{table} [h!]
\centering
\begin{tabular}{|c|c| c| c| c | c |}
\hline
Test Cases & R1 & R2 & R3& R4& NFR1  \\
 \hline   
 Valid & X&X & X&X &X  \\
 \hline
  Invalid force   & X& X& & &  \\
 \hline
  Invalid Joints   &X & X& & &  \\
 \hline
  Invalid Members &X &X & & &  \\
 \hline
 Invalid supports &X &X & & &  \\
 \hline
 Invalid index &X &X & & &  \\
 \hline
 Invalid Connection &X &X & & &  \\
 \hline
 Invalid Location &X &X & & &  \\
 \hline
  Valid n4m5  X&X &X &X &X &X  \\
 \hline
  Valid n5n7  X& X& X& X&X &X  \\
 \hline
  Valid Crane X & X& X& X& X& X \\
 \hline
 
\end{tabular}
\caption{\label{tbl_tr_TC}Traceability Matrix between tests and requirements.}
\end{table}



\section{Unit Test Description} \label{sec_unitest}

%\wss{Reference your MIS (detailed design document) and explain your overall   philosophy for test case selection.}  
In this section, we explain our overall philosophy of test case selection. The main idea is to divide test cases to two categories: the first one to check input verification module and the second to verify the correctness of the output validation module. for more detailed information about design and modules please check 
\href{https://github.com/Maryamvalian/project741/blob/de6c2a97c2152f912c1ed33a18b693b2e031ab56/docs/Design/SoftDetailedDes/MIS.pdf}{MIS Document}.

\subsection{Unit Testing Scope}
In this project, we use Scipy library of Python to solve our linear equations. Testing this library is not in the scope of this document and we trust its results. For unit testing we use Pytest to automatically check modules.


\subsection{Tests for Functional Requirements} 


\subsubsection{Input parameters Module} \label{t_in}


\begin{enumerate}

\item{Input parameter\\}

Type:  Automatic
					
Initial State: Truss Tool is running.
					
Input: import input.txt
					
Output: assert = True

Test Case Derivation: The parameters inputted by the user should 
match the state variable from Inputs Module. An assert statement
will return True if these are equal.

How test will be performed: with help of Pytest Library.
    
\end{enumerate}

\subsubsection{Calculation Module} \label{t_cal}


\begin{enumerate}

\item{Caclulate react module\\}

Type:  Automatic
					
Initial State: Truss Tool is running.
					
Input: import input.txt
					
Output: assert = True

Test Case Derivation: Test Case Derivation: Outputs of the system should match the values
calculated by an online software \cite{valdivia}. An assert statement
will return True if the mean error between calculated outputs and
expected outputs is less than $0.1$

How test will be performed: with help of Pytest Library.
 \item{Caclulate Internals module\\} 

Type:  Automatic
					
Initial State: Truss Tool is running.
					
Input: import input.txt
					
Output: assert = True

Test Case Derivation: Test Case Derivation: Outputs of the system should match the values
calculated by an online software \cite{valdivia}. An assert statement
will return True if the mean error between calculated outputs and
expected outputs is less than $0.1$

How test will be performed: with help of Pytest Library.   

\end{enumerate}
\subsubsection{Output Module} \label{t_out}


\begin{enumerate}

\item{Outputs\\}

Type:  Automatic
					
Initial State: Truss Tool is running.
					
Input: import input.txt
					
Output: Generates output.txt

Test Case Derivation: Outputs should include all the calculated support reactions and internal forces.

How test will be performed: Manually checking the file output.txt in the Src folder.
    
\end{enumerate}

\subsection{Tests for Nonfunctional Requirements}
Testing nonfunctional requirements of units is not required for Truss. Tests
for nonfunctional requirements of system are introduced in section 5.2.

\subsection{Traceability Between Test Cases and Modules}
The purpose of the traceability matrices is to provide easy references on what
has to be additionally modified if a certain component is changed. Table 
shows the dependencies between the test cases and the all of designed modules except the modules that are implemented by the operating system or Python.


\begin{tabular}{l l} 
  \toprule		
  \textbf{Test Cases} & \textbf{Modules}\\
  \midrule 
  
   \ref{t_in}& Inputs Module\\
  \ref{t_cal} & React Module\\
  \ref{t_cal} & Internal Module\\
  \ref{t_out} & Outputs Module\\

  
\bottomrule
\end{tabular}\\


				
\bibliographystyle{plainnat}

\bibliography{refs/References.bib}

\newpage

\section{Appendix}

For easy understanding, we provide a symbolic parameters table briefly here. For a more detailed table please check SRS Document.

\subsection{Symbolic Parameters}
\begin{table} [h!]
\centering
\begin{tabular}{|c| c |}
\hline
Symbol & Meaning   \\
 \hline   
 $J$& Tuples of the joint's coordination.\\
 &Each tuple is a pair of two-dimensional geometric coordination.   \\
 \hline
 $M$ & Tuples of members. Each tuple is pair of joint index  \\
 \hline
 $F$ & External forces. Tuples of the vertical and\\
 & horizontal amount of force for each joint  \\
 \hline
 $S=(S_{p} ,S_R)$ &Pair of two indices. Index of pinned support\\
 & and index of roller support \\
 \hline

 
\end{tabular}
\caption{\label{tbl_tr_TC} Symbol Table for VnV document.}
\end{table}

\subsection{Usability Survey Questions?}

Not Applicable.

\newpage{}

\end{document}